%% ------------------------------ Abstract ---------------------------------- %%
\begin{abstract}
  Renewed interest in advanced nuclear power reactors such as the Versatile Test 
  Reactor (VTR) at Idaho National Laboratory (INL) has encouraged enhanced 
  modeling of Fast Reactors (FRs). Since their inception in the 
  early days of nuclear engineering with reactors such as at the Experimental 
  Breeder Reactor (EBR-I) and Fermi 1, many new modeling techniques have been 
  developed. This work seeks to introduce modern methods to the simulation of 
  FRs.

  In this work, the multigroup neutron diffusion equation is solved via the
  Finite Element Method (FEM). This method allows for the use of unstructured
  and general meshes. By using an unstructured mesh, physical phenomena such as
  thermal expansion can be modeled and used to distort the mesh. Additionally,
  the FEM will allow for simplified spatial refinement by means of both 
  mesh refinement and the use of higher-order methods without regenerating the 
  mesh. Unique to this thesis is the use of pentahedral wedge elements in the
  FEM mesh. Wedge elements are selected for their natural description of
  hexagonal geometry common to FRs.

  Thermal feedback effects within a FR are also modeled in this work. A
  simplified thermal hydraulic model is used, modeling both axial heat 
  convection and radial heat conduction. Temperatures from this model are used 
  to calculate on-the-fly, temperature dependent neutron cross-sections to 
  capture reactivity feedback effects. Additionally, a thermal expansion 
  model is used to simulate the thermal expansion of fuel and structural 
  components within the reactor.  These effects have been demonstrated to 
  significantly impact reactor behavior in experiments such as those performed 
  at Experimental Breeder Reactor II (EBR-II).

  Using these models, a typical FR is simulated at operating conditions. The 
  models as implemented demonstrate expected reactor behavior for an FR. For the
  simulated reactor, reactivity feedback coefficients are calculated which would
  not be possible without a coupled multiphysics model. These results can
  be used to visualize the inherent safety features and feedback effects of such 
  a nuclear reactor.
  
\end{abstract}

%% ---------------------------- Copyright page ------------------------------ %%
\makecopyrightpage

%% -------------------------------- Title page ------------------------------ %%
\maketitlepage

%% -------------------------------- Dedication ------------------------------ %%
\begin{dedication}
  \centering To the future of clean electricity.
\end{dedication}

%% -------------------------------- Biography ------------------------------- %%
\begin{biography}
  William C. Dawn was born and raised in Stafford, Virginia. He attended public
  schools there for his primary education, participating in the Commonwealth
  Governor's School in High School. William earned a Bachelor's of Science 
  degree in Nuclear Eneingeering from North Carolina State University (NC State)
  in May 2017. After his Master's degree, William will remain at NC State to 
  pursue a Ph.D. degree in Nuclear Engineering. 

  William is a fellow of the Integrated University Program (IUP) facilitated by
  U.S. Department of Energy Office of Nuclear Energy (DOE-NE). During his 
  undergraduate and graduate careers, he has had the opportunity to work with 
  GE-Hitachi Nuclear Energy Americas LLC (GEH) and Oak Ridge National
  Laboratory (ORNL). William has also made contributions to the Consortium for
  Advanced Simulation of LWRs (CASL).
\end{biography}

%% ----------------------------- Acknowledgements --------------------------- %%
\begin{acknowledgements}
  This work would not have been possible without the help of friends and family.
  I would like to thank my Mom and Dad, Suzanne and Bill Dawn, for their
  patience, their listening, and their advice. Their support has helped to make
  this work a reality.

  I would also like to thank my advisor, Dr. Scott Palmtag. We have both learned
  tremendously during this process. His consistency and desire to know more have
  kept me busy these last few months and I am grateful.
\end{acknowledgements}

%% -------------------------------- Disclaimer ------------------------------ %%
\clearpage
\vspace*{\fill}
\begin{center}
  \mbox{\parbox{5in}{
  This material is based upon work supported under an Integrated
  University Program Graduate Fellowship. Any opinions, findings, conclusions,
  or recommendations expressed in this publication are those of the author and
  do not necessarily reflect the views of the Department of Energy Office of
  Nuclear Energy.
  }}
\end{center}
\vspace*{\fill}
\clearpage

\thesistableofcontents

\thesislistoftables

\thesislistoffigures
