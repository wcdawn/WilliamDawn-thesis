\chapter{FINITE ELEMENT NEUTRON DIFFUSION}
\label{ch:neutronDiffusion}

\section{Introduction}
  For typical nuclear reactor applications, diffusion theory well approximates 
  the neutron distribution within the reactor. The neutron diffusion equation is
  a second order partial differential equation in space and energy. In standard
  notation, the continuous neutron diffusion equation is presented as
  \begin{multline}\label{eq:continuous_diffusion}
    -\grad \cdot (D(\vr,E) \grad \phi(\vr,E)) + \Sigma_t(\vr,E) \phi(\vr,E) = \\
      \frac{\chi(\vr,E)}{\keff} \int_0^{\infty} \nu_f(\vr,E') \Sigma_f(\vr,E') 
      \phi(\vr,E') \; dE' + \int_0^{\infty} \Sigma_s(\vr,E' \rightarrow E) 
      \phi(\vr,E') \; dE'
  \end{multline}
  Where $D$ is the diffusion constant, $\phi$ is the scalar neutron flux, 
  $\Sigma_t$ is the total cross section, $\chi$ is the fission neutron 
  spectrum, $\keff$ is the effective neutron multiplication factor, $\Sigma_f$ 
  is the fission cross section, $\nu_f$ is the neutron yield per fission, and 
  $\Sigma_s(\vr,E' \rightarrow E)$ is the scattering cross section for neutrons 
  at position $\vr$ scattering from energy $E'$ to $E$.
  
  The neutron diffusion equation must be discretized in space and 
  energy to be solved numerically. Energy discretization is relatively 
  straight-forward and is performed using the multigroup method. Spatial 
  discretization requires more attention and will be done with the Finite 
  Element Method (FEM). This method is selected for several reasons. It allows
  for easily increasing the order of the method by increasing the order of the 
  elements with no changing of the mesh required. Coordinates of nodes can be 
  easily updated to reflect physical phenomena such as thermal expansion 
  (\chref{ch:thermalExpansion}). Additionally, material properties are 
  calculated on an element basis allowing for fine detail updates to the 
  material properties during the calculation.
  
  For energy discretization, An energy structure is described as $\{E_g\}$ for 
  $g = 1,2,\ldots,G$ and by convention
  \[ E_G > E_{G-1} > \ldots > E_2 > E_1  \]
  Then, multigroup constants can be calculated based on the energy group 
  structure and the known cross sections. Multigroup constants are calculated to
  preserve the number of neutrons produced or destroyed. That is, the 
  calculation preserves reaction rates where the reaction rate for reaction $x$
  is defined as $R_x=\Sigma_x \phi$. Multigroup constants are then calculated. A
  formal derivation is given in \cite{duderstathamilton} and the results are 
  presented below.
  \begin{align}
    D_g(\vr) &= \frac{\int_{E_g}^{E_{g-1}} D(\vr,E') \grad \phi(\vr,E)\;dE}
      {\int_{E_g}^{E_{g-1}} \grad \phi(\vr,E)\;dE} \\
    \Sigma_{t,g}(\vr) &= \frac{\int_{E_g}^{E_{g-1}} \Sigma_t(\vr,E) 
      \phi(\vr,E)\;dE}{\int_{E_g}^{E_{g-1}} \phi(\vr,E)\;dE} \\
    \nu\Sigma_{f,g}(\vr) &= \frac{\int_{E_g}^{E_{g-1}} \nu_f(\vr,E)
      \Sigma_f(\vr,E) \phi(\vr,E)\;dE}{\int_{E_g}^{E_{g-1}} 
      \phi(\vr,E)\;dE} \\
    \Sigma_{s,g\rightarrow g'}(\vr) &= \frac{\int_{E_g'}^{E_{g'-1}} 
      \int_{E_g}^{E_{g-1}} \Sigma_s(\vr,E' \rightarrow E) \phi(\vr,E')\;dE\;dE'}
      {\int_{E_g}^{E_{g-1}} \phi(\vr,E)\;dE}  \\
    \chi_g(\vr) &= \int_{E_g}^{E_{g-1}} \chi(\vr,E) \; dE \\
    \phi_g(\vr) &= \int_{E_g}^{E_{g-1}} \phi(\vr,E) \; dE
  \end{align}
  Note that cross sections $\nu_f(\vr,E)$ and $\Sigma_f(\vr,E)$ have been 
  combined. This is necessary and a mathematical result of the group collapse.
  Then, \eref{eq:continuous_diffusion} can be discretized in energy as
  \begin{align}\label{eq:multigroup_diffusion}
    - \grad \cdot ( D_g(\vr) \grad \phi_g(\vr)) + \Sigma_{t,g}(\vr) \phi_g(\vr)= 
      \frac{\chi_g(\vr)}{\keff} \sum_{g'=1}^{G} \nu\Sigma_{f,g'}(\vr) 
      \phi_{g'}(\vr) + \sum_{g'=1}^{G} \Sigma_{s,g' \rightarrow g}(\vr) 
      \phi_{g'}(\vr)
  \end{align}
  The neutron diffusion equation has now been discretized in energy.
  Spatial dicretization will be based on the Finite Element Method (FEM) and 
  will be discussed in \sref{sec:formulation:derivation}.
  
  % todo this discussion may need to be moved to 'Implementation'
  The total neutron cross section includes the contribution due to 
  self-scattering. That is, due to $\Sigma_{s,g\rightarrow g}$. This can be 
  removed from \eref{eq:multigroup_diffusion} for simplicity.
  \begin{equation} \label{eq:multigroup_removal}
    - \grad \cdot( D_g(\vr) \grad \phi_g(\vr)) + \Sigma_{r,g}(\vr) \phi_g(\vr) = 
      \frac{\chi_g(\vr)}{\keff} \sum_{g'=1}^{G} \nu\Sigma_{f,g'}(\vr) 
      \phi_{g'}(\vr) + \sum_{g'=1, g' \ne g}^{G} \Sigma_{s,g' \rightarrow g}(\vr) 
      \phi_{g'}(\vr)
  \end{equation}
  Where $\Sigma_{r,g}$ is the removal cross section and $\Sigma_{r,g}(\vr) = 
  \Sigma_{t,g}(\vr) - \Sigma_{s,g\rightarrow g}(\vr)$. For simplicity, the
  neutron sources in \eref{eq:multigroup_removal} can be  combined into a
  single term.
  \begin{equation} \label{eq:multigroup_source}
    - \grad \cdot( D_g(\vr) \grad \phi_g(\vr)) + \Sigma_{r,g}(\vr) \phi_g(\vr) = 
      q_g(\vr)
  \end{equation}
  Where $q_g(\vr)$ is the combined neutron source at position $\vr$. $q$ can 
  then be further divided into contributions due to fission ($q_{fiss}$), up-
  scattering ($q_{up}$) when a neutron increases in energy, and down-scattering 
  when a neutron decreases in energy ($q_{down}$).
  \begin{align}
    q_g(\vr) &= q_{fiss,g}(\vr) + q_{up,g}(\vr) + q_{down,g}(\vr) \\
    q_{fiss,g}(\vr) &= \frac{\chi_g(\vr)}{\keff} \sum_{g'=1}^{G} 
      \nu \Sigma_{f,g'}(\vr) \phi_{g'}(\vr) \\
    q_{up,g}(\vr) &= \sum_{g'=g+1}^{G} \Sigma_{s,g' \rightarrow g}(\vr)
      \phi_{g'}(\vr) \\
    q_{down,g}(\vr) &= \sum_{g'=1}^{g} \Sigma_{s,g' \rightarrow g}(\vr)
      \phi_{g'}(\vr)
  \end{align}
  Where the difference between $q_{up}$ and $q_{down}$ are the limits of the 
  summation. This form allows for operator splitting of the neutron source term.
  In an iterative scheme, it will be necessary for fission and up-scatter 
  sources to use a different flux iterate than down-scatter so this division
  will prove useful.
  

\section{Formulation}
  \subsection{Derivation}
    \label{sec:formulation:derivation}
    The only remaining continuous variable in the problem is the spatial 
    variable $\vr$. This will be discretized according to the Finite Element 
    Method (FEM). The form of the diffusion equation to be discretized is 
    \eref{eq:multigroup_source}. The problem is solved in a finite domain 
    $\vr \in \Omega$ where $\partial \Omega$ represents the boundary of the 
    domain where some boundary condition is specified. Boundary condition 
    options provided include
    \begin{enumerate}
      \item Mirror. $\grad \phi_g(\vr) = 0$ for $\vr \in \partial \Omega$.
      \item Albedo. $D_g(\vr) \grad \phi_g(\vr) + \albedo \phi_g(\vr)=0$ for 
        $\vr \in \partial \Omega$ where $\albedo$ is a real constant specified
        by the user. For vacuum conditions, $\alpha = \half$.
      \item Zero Flux. $\phi_g(\vr) = 0$ for $\vr \in \partial \Omega$.
    \end{enumerate}
    (Note: the order of the above list corresponds to the order of boundary 
    condition precedent in code with the greater the integer, the greater the 
    precedent.)
    
    Finite Element derivation begins with \eref{eq:multigroup_source}.
    The equation is multiplied by a testing function $v(\vr) \in H_1(\Omega)$ 
    Where $H$ is the Sobolev Space. Then, the equation is integrated over the 
    problem domain. This yields the Weak Form or Variational Form of the 
    problem.
    \begin{align}
      -\grad \cdot (D_g(\vr) \grad \phi_g(\vr)) + \Sigma_{r,g}(\vr) \phi_g(\vr)
        &=q_g(\vr) \\
      -\grad \cdot (D_g(\vr) \grad \phi_g(\vr)) v(\vr) + 
        \Sigma_{r,g}(\vr) \phi_g(\vr) v(\vr) &=
        q_g(\vr) \\
      - \int_{\Omega} \grad \cdot (D_g(\vr) \grad \phi_g(\vr)) v(\vr) \; d\Omega
        \int_{\Omega} \Sigma_{r,g}(\vr) \phi_g(\vr) v(\vr) \;d\Omega &=
        \int_{\Omega} q_g(\vr) v(\vr) \;d\Omega
    \end{align}
    
    For the purposes of this application, material cross sections and the
    neutron source are assumed to be constant within an element. Then, the 
    integral can be partitioned into a sum of integrals over the elements in the
    domain assuming the set of elements $\{\Omega_e\} = \Omega$ for 
    $e = 1,2,\ldots,E$ where $E$ is the total number of elements.
    \begin{equation} \label{eq:element_by_element}
      -\sum_{e=1}^{E} D_{g,e} 
        \int_{\Omega_e} \grad \cdot \grad \phi_g(\vr) v(\vr) \; d\Omega_e +
        \sum_{e=1}^{E} \Sigma_{r,g,e} \int_{\Omega_e} \phi_g(\vr) v(\vr) 
        \;d\Omega_e = \sum_{e=1}^{E} q_{g,e} \int_{\Omega_e} v(\vr) 
        \; d\Omega_e
    \end{equation}
    The Second Green's Theorem is used to simplify the integral in the first
    term. A proof can be found in \cite{textbookli} in Theorem 9.2.
    \begin{equation} \label{eq:greens}
      -\int_{\Omega_e} \grad \cdot \grad \phi_g(\vr) v(\vr) \;d\Omega_e =
        -\int_{\partial \Omega_e}  
        \frac{\partial \phi_g(\vr)}{\partial \vn} v(\vr)\; ds + \int_{\Omega_e} 
        \grad \phi_g(\vr) \cdot \grad v(\vr) \; d\Omega_e
    \end{equation}
    Where $\frac{\partial \phi_g(\vr)}{\partial \vn}$ is the outward normal 
    derivative and the integral $ds$ is a line integral in two dimensions or a 
    surface integral in three dimensions. Recognizing that this quantity will 
    only be relevant on the boundary of the problem, the value of the outward 
    normal derivative may be specified in a boundary condition. Specifically, 
    the Albedo boundary condition which has the following form for $\vr \in 
    \partial \Omega$. 
    \begin{align}
      D_g(\vr) \grad \phi_g(\vr) + \albedo \phi_g(\vr) &= 0 \\
      \grad \phi_g(\vr) &= \frac{-\albedo \phi_g(\vr)}{D_g}
    \end{align}
    Substituting \eref{eq:greens} into  \eref{eq:element_by_element} and 
    assuming the outward normal derivative is specified in the form of an Albedo
    boundary condition.
    \begin{multline} 
      -\sum_{e=1}^{E} D_{g,e} \int_{\partial \Omega_e} v(\vr) 
        \frac{\partial \phi_g(\vr)}{\partial \vn} \;ds + \sum_{e=1}^{E} D_{g,e}
        \int_{\Omega_e} \grad \phi_g(\vr) \cdot \grad v(\vr) \; d\Omega_e + \\
        \sum_{e=1}^{E} \Sigma_{r,g,e} \int_{\Omega_e} \phi_g(\vr) v(\vr) 
        \; d\Omega_e =
        \sum_{e=1}^{E} q_{g,e} \int_{\Omega_e} v(\vr) \; d\Omega_e
    \end{multline}
    \begin{multline} \label{eq:element_boundary}
      \sum_{e=1}^{E} \albedo \int_{\partial \Omega_e} v(\vr) 
        \phi_g(\vr) \;ds + \sum_{e=1}^{E} D_{g,e}
        \int_{\Omega_e} \grad \phi_g(\vr) \cdot \grad v(\vr) \; d\Omega_e + \\
        \sum_{e=1}^{E} \Sigma_{r,g,e} \int_{\Omega_e} \phi_g(\vr) v(\vr) 
        \; d\Omega_e =
        \sum_{e=1}^{E} q_{g,e} \int_{\Omega_e} v(\vr) \; d\Omega_e
    \end{multline}
    Now the function of interest $\phi_g(\vr)$ is assumed to be a linear 
    combination of chosen basis functions $\{\basis_i\}$.
    \begin{equation} \label{eq:linear_combination}
      \phi_g(\vr) = \sum_{i=1}^{N} \alpha_{g,i} \basis_i(\vr)
    \end{equation}
    Where coefficients $\{\alpha_i\}$ are unknown and will be determined. 
    Typically, these basis functions have unit magnitude and are centered at the
    node  points so the coefficients $\alpha_i$ are the approximated solution at
    the nodes. Basis functions are typically polynomials of arbitrary magnitude. 
    Linear and quadratic polynomials are common but for the application 
    presented here, only linear basis functions are explored.
    The test function $v(\vr)$ is also chosen as a linear combination of the 
    basis functions.
    \begin{equation} \label{eq:linear_superposition}
      v(\vr) = \sum_{j=1}^{N} \basis_j(\vr)
    \end{equation}
    The testing function is arbitrary so the magnitude is fixed to the magnitude
    of the basis function.
    
    \eref{eq:linear_combination} and \eref{eq:linear_superposition} are plugged
    into \eref{eq:element_boundary}. This yields a linear system of equations.
    \begin{multline}
      \sum_{e=1}^{E} \albedo \sum_{i=1}^{N} \alpha_{i,g}
        \int_{\partial \Omega_e}
        \basis_i(\vr)  \basis_j(\vr) \;ds +
        \sum_{e=1}^{E} D_{g,e} \sum_{i=1}^{N} \alpha_{i,g}
        \int_{\Omega_e} \grad \basis_i(\vr) \cdot \grad \basis_i(\vr)\;d\Omega_e
        + \\
        \sum_{e=1}^{E} \Sigma_{r,g,e} \sum_{i=1}^{N} \alpha_{i,g}
        \int_{\Omega_e} \basis_i(\vr) \basis_i(\vr) \; d\Omega_e =
        \sum_{e=1}^{E} q_{g,e} \sum_{i=1}^{N} 
        \int_{\Omega_e} \basis_i(\vr) \; d\Omega_e
    \end{multline}
    \begin{multline}
      \sum_{i=1}^{N} \alpha_{i,g} \sum_{j=1}^{N} \left(
        \sum_{e=1}^{E} \albedo \int_{\partial \Omega_e}
        \basis_i(\vr)  \basis_j(\vr) \;ds +
        \sum_{e=1}^{E} D_{g,e} 
        \int_{\Omega_e} \grad \basis_i(\vr) \cdot \grad \basis_j(\vr)\;d\Omega_e
        \right.
        + \\
        \left.
        \sum_{e=1}^{E} \Sigma_{r,g,e}
        \int_{\Omega_e} \basis_i(\vr) \basis_j(\vr) \; d\Omega_e \right) =
        \sum_{i=1}^{N} \left(
        \sum_{e=1}^{E} q_{g,e} 
        \int_{\Omega_e} \basis_i(\vr) \; d\Omega_e \right)
    \end{multline}
    \begin{align}
      a(\basis_i,\basis_j) &= f(\basis_i) \\
      \ma \vu &= \vf
    \end{align}
    $a(\basis_i,\basis_j)$ is the bilinear form and $f(\basis_i)$ is the linear 
    form of the finite element system. In matrix notation, $\ma$ is described 
    by the integral quantities (more to follow),  $\vu = \{\alpha_i\}$, and
    $\vf$ is described by the source integral quantity.
    
    Inspecting the matrix $\ma$ and the vector $\vf$ reveals the following.
    \begin{align}
      \label{eq:matrix_population}
      A_{i,j,g,e} &= \albedo \int_{\partial \Omega_e} \basis_i(\vr) 
        \basis_j(\vr) \; ds + D_{g,e} 
        \int_{\Omega_e} \grad \basis_i(\vr) \cdot \grad \basis_j(\vr) \;
        d\Omega_e + \Sigma_{r,g,e} \int_{\Omega_e} \basis_i(\vr) \basis_j(\vr)
        \; d\Omega_e \\
      \label{eq:vector_population}
      f_{i,g,e} &= q_{g,e} \int_{\Omega_e} \basis_i(\vr) \;d\Omega_e
    \end{align}
    Then, 
    \begin{align}
      A_{i,j,g} &= \sum_{e=1}^{E} A_{i,j,g,e} \\
      f_{i,g} &=  \sum_{e=1}^{E} f_{i,g,e}
    \end{align}
    which leads to the natural population of the matrix $\ma$ on an 
    element-by-element basis. That is, the matrix $\ma$ is assembled by looping
    through all of the elements and summing their contribution to the matrix. 
    Note that the contribution due to the surface integral will be zero in 
    elements not on the boundary and may also be zero for problems with select
    boundary conditions. The population of the vector $\vf$ is done similarly. 
    Then, the matrix $\ma$ and the vector $\vf$ are known for each energy group.
    The equations are solved on group at a time and $\phi_g$ is calculated and 
    stored.
    
    Though the notation may be obtuse, the above reduces to a linear system of
    equations. These equations are constructed from the integral quantities 
    specified by the FEM and the coefficients given by the cross sections and
    fixed source regions. The integral quantities themselves are expressed 
    explicitly in the next section.
    
  \subsection{Matrix Quantities}
    For certain simple elements, the integral quantities described in 
    \eref{eq:matrix_population} and \eref{eq:vector_population} have exact 
    analytic forms. The for this application, linear triangles and linear wedges
    are investigated and many of the integrals have exact expressions. If these 
    quantities cannot be expressed exactly or doing so would be computationally
    difficult, quadratures can be used and for certain problems, these 
    quadratures can express the integrals exactly. This will be discussed in 
    \sref{sec:quadratures}.
    \subsubsection{Linear Triangles}
      Linear triangles are common to two dimensional finite element methods and
      have been investigated in many applications \cite{Hosseini2017} 
      \cite{Hosseini2013} \cite{Hosseini2015}. This is a triangle  defined by
      three corner coordinates with basis functions located on each corner
      \cite{vtk}. The reference triangle $T_{ref}$ is located in
      $\xi \in [0,1]$ and $\eta \in [0,1-\xi]$. The basis functions for the 
      reference linear triangle are provided.
      \begin{align}
        \basis_1(\xi,\eta) &= \xi \\
        \basis_2(\xi,\eta) &= \eta \\
        \basis_3(\xi,\eta) &= 1-\xi-\eta
      \end{align}
      
      Originally proposed in \cite{textbookwhite}, there are simple expressions
      for the quantities. The expression for the line integral is found in 
      \cite{computerLab}. For a triangle with corners $\{ x_i,y_i \}$.
      \begin{align}
        \int_{\Omega_e} \grad \basis_i(\vr) \cdot \grad \basis_j(\vr) 
          \;d\Omega_e &= \frac{1}{4 A_e}
          ((x_{i+1}-x_{i+2})(x_{j+1}-x_{j+2}) + 
          (y_{i+1}-y_{i+2})(y_{j+1}-y_{j+2})) \\
        \int_{\Omega_e} \basis_i(\vr) \basis_j(\vr) \;d\Omega_e &= 
          \frac{A_e}{12} (1+\delta_{ij}) \\
        \int_{\Omega_e} \basis_i(\vr) \;d\Omega_e &= \frac{A_e}{3} \\
        \int_{\partial \Omega_e} \basis_i(\vr) \basis_j(\vr) \;ds &=
          \frac{L_e}{6}(1+\delta_{ij}) 
      \end{align}
      Where $A_e$ is the area of the triangular element $L_e$ is the length of 
      the edge between node $i$ and node$j$, and $\delta_{ij}$ is the Kronecker
      delta such that
      \begin{equation} \label{eq:kroneker_delta}
        \delta_{ij} =
        \begin{cases}
          0 & \text{if } i \ne j \\
          1 & \text{if } i = j
        \end{cases}
      \end{equation}
      For higher order triangular elements, it will be necessary to employ a 
      quadrature.
    \subsubsection{Linear Wedges}
      Wedge elements have not reached the same commonality as triangular 
      elements but are extruded triangles. Geometrically, the shape is a 
      pentahedron or a triangular prism. However, their exact geometric
      relation is not fixed and the nodes are free to expand. These shapes 
      are unique because three of the faces are quadrilateral and two of
      the faces are triangular. The reference wedge $W_{ref}$ is located in 
      $\xi \in [0,1]$, $\eta \in [0,1-\xi]$, and $\zeta \in [-1,1]$. The basis
      functions for the reference wedge are provided.
      \begin{align}
        \basis_1(\xi,\eta,\zeta) &= \half (1-\zeta)(1-\xi-\eta) \\
        \basis_2(\xi,\eta,\zeta) &= \half (1-\zeta)\xi \\
        \basis_3(\xi,\eta,\zeta) &= \half (1-\zeta)\eta \\
        \basis_4(\xi,\eta,\zeta) &= \half (1+\zeta)(1-\xi-\eta) \\
        \basis_5(\xi,\eta,\zeta) &= \half (1+\zeta)\xi \\
        \basis_6(\xi,\eta,\zeta) &= \half (1+\zeta)\eta 
      \end{align}
      The values presented herein are not yet found in literature and are 
      calculated by there author and published here for the first time.
      \begin{align}
        \int_{\Omega_e} \basis_i(\vr) \basis_j(\vr) \;d\Omega_e &= 
          \frac{V_e}{2}
          \begin{pmatrix}
            \frac{1}{18} & \frac{1}{36} & \frac{1}{36} & \frac{1}{36} & 
              \frac{1}{72} & \frac{1}{72} \\
            \frac{1}{36} & \frac{1}{18} & \frac{1}{36} & \frac{1}{72} & 
              \frac{1}{36} & \frac{1}{72} \\
            \frac{1}{36} & \frac{1}{36} & \frac{1}{18} & \frac{1}{72} & 
              \frac{1}{72} & \frac{1}{36} \\
            \frac{1}{36} & \frac{1}{72} & \frac{1}{72} & \frac{1}{18} & 
              \frac{1}{36} & \frac{1}{36} \\
            \frac{1}{72} & \frac{1}{36} & \frac{1}{72} & \frac{1}{36} & 
              \frac{1}{18} & \frac{1}{36} \\
            \frac{1}{72} & \frac{1}{72} & \frac{1}{36} & \frac{1}{36} & 
              \frac{1}{36} & \frac{1}{18} 
          \end{pmatrix} \\
        \int_{\Omega_e} \basis_i(\vr) \;d\Omega_e &= \frac{V_e}{12} \\
        \int_{\partial \Omega_e} \basis_i(\vr) 
          \basis_j(\vr) \;d\partial\Omega_e &= 
          \begin{cases}
            \frac{A_{\Delta}}{12}(1+\delta_{ij}) & \text{if triangle} \\
            \frac{A_{\Box}}{36}(1+\delta_{ij})(1-\half \delta_{i,(5-j)}) &
              \text{if quadrilateral}
          \end{cases}
      \end{align}
      Where $V_e$ is the volume of the element. This matrix is indexed $A_{ij}$
      and is presented as a matrix because of its irregular form. Notice the 
      integral containing the gradient operator has been omitted because if 
      it could be computed analytically, it would be less computationally 
      efficient than using a quadrature.
      
  \subsection{Quadratures}
    \label{sec:quadratures}
    Quadratures are sets of coordinates and weights which allow for the exact 
    integration of polynomials of given order. For a given set of weights 
    $\{w_i\}$ and a set of coordinates $\{\vx\}$, the quadrature can be 
    expressed as follows.
    \begin{equation}
      \int_{\Omega} f(\vx) \;d\Omega \approx \sum_{i=1}^{N} w_i f(\vx_i)
    \end{equation}
    where $\Omega$ is an arbitrary domain described by $\{\vx_i\}$. The above
    quadrature will exactly integrate a polynomial of the order of the 
    quadrature. It is not necessarily true that $N$ be the order of the 
    quadrature.
    
    For one dimensional integrals, the Gaussian quadrature is common and the 
    most compact quadrature. The Gaussian quadrature exactly integrates a 
    polynomial of order $n$ using exactly $n$ points. Weights and coordinates
    for this quadrature set are presented in \cite{gaussianQuadrature}. For this
    quadrature, $n=N$.
    
    Two dimensional and three dimensional quadratures are necessary for the 
    solution of this problem. Triangular quadratures are not as simply derived 
    as line quadratures and the number of points need not equal the order of the
    polynomial integrated. The triangular quadrature as implemented here is 
    symmetric and open. That is, there are no points on the boundary of the 
    triangle. The structure for this triangular quadrature is found in 
    \cite{triangleQuadrature}.
    
    Quadrilateral quadrature sets are simply tensor products of two line 
    Gaussian quadratures. For an order $N$ polynomial, now $n^2$ points are 
    required. 
    
    Wedge quadrature sets are simply tensor products of a line Gaussian 
    quadrature and a triangular quadrature. 
    
    Basis functions are polynomials of first, second, or third order. These 
    quadratures are capable of exactly integrating functions of given order so 
    there is a quadrature order that will exactly integrate the Finite Element 
    quantities to the precision of the quadrature and the numeric precision. The
    table of the order required for exact integration are provided in 
    \tref{tab:quadrature_orders}.
    \begin{table}
      \caption{Quadrature orders for FEM quantities.}
      \label{tab:quadrature_orders}
      \begin{center}
        \begin{tabular}{lccc}
          \toprule
          Quantity & Linear & Quadratic & Cubic \\
          \midrule
          $\int_{\Omega} \basis_i(\vr) \;d\Omega$ & 1 & 2 & 4\footnotemark \\
          $\int_{\Omega} \basis_i(\vr) \basis_j(\vr) \;d\Omega$ &
            2 & 4 & 6 \\
          $\int_{\Omega} \grad \basis_i(\vr) \cdot \grad \basis_j(\vr) 
            \;d\Omega$ & 2 & 3 & 5 \\
          \bottomrule
        \end{tabular}
      \end{center}
    \end{table}
    \footnotetext{A third order quadrature would be exact but the quadrature
    would have negative weights so a fourth order quadrature is 
    selected.}
    
    All of the quadratures described here are tabulated for a reference element
    be it a line, a triangle, a quadrilateral, or a wedge. Integration in the 
    FEM is performed on an arbitrary element in space. Therefore, it is 
    necessary to perform a coordinate transform when using a quadrature set.
    \begin{equation}
      \int_{\Omega} f(\vx) \;d\Omega = 
        \int_{\Omega_{ref}} f(\vx) \lvert \mj \rvert \;d\Omega \approx
        \sum_{i=1}^{N} w_i f(\vx_i) \lvert \mj_i \rvert
    \end{equation}
    where $\mj$ is the Jacobian matrix, $\mj_i$ is the Jacobian matrix at 
    quadrature coordinate $\vx_i$, and $\lvert \cdot \rvert$ represents
    the matrix determinant. Notationally, $J=\lvert \mj \rvert$ and is termed
    the Jacobi.
    
    For simple elements, the Jacobi is constant over the element and can be
    precalculated to save from allocating, populating, and taking the 
    determinant of a matrix for each integration point. For the elements of 
    concern, these values are presented in \tref{tab:jacobi} as found in 
    \cite{textbookcolorado}.
    \begin{table}
      \caption{Jacobi for selected elements}
      \label{tab:jacobi}
      \begin{center}
        \begin{tabular}{ll}
          \toprule
          Element & $J$ \\
          \midrule
          Triangle      & $A_e$ \\
          Quadrilateral & $\frac{1}{4} A_e$ \\
          Wedge         & $\half V_e$ \\
          \bottomrule
        \end{tabular}
      \end{center}
    \end{table}

\section{Implementation}
  A Finite Element neutron diffusion solution method has been developed using 
  the above formulae. The program begins with a geometry description specified
  in a plaintext VTK file. Cross sections are specified in either a plaintext
  user format or the ISOTXS format as common to fast reactor applications 
  and the multigroup cross section generator \mcc. The multigroup neutron 
  diffusion equation is then solved according to \eref{eq:multigroup_source}.
  The resulting effective neutron multiplication factor, $\keff$, is written to
  an output file. The multigroup neutron flux is written to a different results
  VTK file for easy visualization.
  \subsection{Algorithm}
    The algorithm for this solution to the diffusion equation is similar to most
    implementations of the multigroup method. The steps unique to the Finite
    Element method are steps \ref{state:fem_matrix} and \ref{state:fem_vector}.
    These require the quantities previously derived and form the linear system 
    described by the FEM. 
    
    In step \ref{state:rcm} the matrix is reordered. Mathematically this has no
    effect on the result as the solution is the same. 
    \begin{algorithm}
      \caption{General Iteration Scheme}
      \begin{algorithmic}[1]
      \State Initialize $\phiavg^{(0)}$.
      \State Order the nodes of the mesh into RCM order. \label{state:rcm}
      \State Calculate $\Sigma_s$, $\Sigma_t$, and $\nu \Sigma_f$ for each 
        element.
      \State Calculate Finite Element Matrix $\ma_g$ for each group. Store this. 
        \label{state:fem_matrix}
      \While{Power Iteration}
      	\State Update the iteration counter. $s=s+1$
      	\State Update $q_{fiss}$ and $q_{up}$ from previous data
          $\phiavg^{(s-1)}$.
        \State Update $\chi$ in each element using previous data
        \For{$g=1,G$}
          \State Update $q_{down}$ from current data $\phiavg^{(s)}$
          \State Calculate total effective source in each element.
          \State Update Finite Element Vector $\vf_g$ with new source.
            \label{state:fem_vector}
          \State Solve $\ma \vu = \vf$ using an iterative technique.
          \State Parse $\vu$ for $\phi$ nodal solution.
          \State Calculate element-average $\phiavg$.
        \EndFor
        \State Update $k_{eff}$.
        \State Check convergence.
      \EndWhile
      \end{algorithmic}
    \end{algorithm}
    
    Note that the Finite Element vector $\vf$ must be updated on each iteration
    of the solution whereas the matrix $\ma$ is described entirely by geometry 
    and the material cross sections. For this reason, $\ma$ can be generated 
    once at the beginning of the problem and stored for the duration of the 
    calculation.
  	% in addition, make sure to mention the matrix is in RCM order
    \FloatBarrier % make sure the algorithm is in the correct section
  \subsection{Memory and Storage}
    The Finite Element matrix $\ma$ is large and sparse so a sparse storage 
    implementation is required. \cite{rcm}
  	% discuss two-table method
    % mention other alternatives
  \subsection{Linear System Solution}
  	% Dr. Kelly's book and CG Method
    % maybe discuss SPD here

\section{Reference Results}
	% develop a format to compare results for these so if updates happen
    % the csv or whatever can simply be updated
  \subsection{Triangular Element Manufactured Solutions}
    \subsubsection{One Dimension, One-Group, Fixed Source}
    \subsubsection{One Dimension, One-Group, Criticality}
    \subsubsection{One Dimension, Two-Group, Criticality}
    \subsubsection{One Dimension, One-Group, Two-Region, Criticality}
    \subsubsection{Two Dimension, One-Group, Criticality}
  \subsection{Two Dimension Reactors}
    \subsubsection{VVER440}
    \subsubsection{SNR}
    \subsubsection{HWR}
    \subsubsection{IAEA}
      % nore0125
      % nore0500
      % refl0125
      % refl0500
  \subsection{Wedge Element Manufactured Solution Finite Cylinder}
  \subsection{Three Dimension Reactors}
    \subsubsection{MONJU}
    \subsubsection{KNK}
