\chapter{FINITE ELEMENT NEUTRON DIFFUSION}
\label{ch:neutronDiffusion}

\section{Introduction}
  For typical nuclear reactor applications, diffusion theory well approximates 
  the neutron distribution within the reactor. The neutron diffusion equation is
  a second order partial differential equation in space and energy. In standard
  notation, the continuous neutron diffusion equation is presented as
  \begin{multline}\label{eq:continuous_diffusion}
    -\grad \cdot (D(\vr,E) \grad \phi(\vr,E)) + \Sigma_t(\vr,E) \phi(\vr,E) = \\
      \frac{\chi(\vr,E)}{\keff} \int_0^{\infty} \nu_f(\vr,E') \Sigma_f(\vr,E') 
      \phi(\vr,E') \; dE' + \int_0^{\infty} \Sigma_s(\vr,E' \rightarrow E) 
      \phi(\vr,E') \; dE'
  \end{multline}
  Where $D$ is the diffusion constant, $\phi$ is the scalar neutron flux, 
  $\Sigma_t$ is the total cross section, $\chi$ is the fission neutron 
  spectrum, $\keff$ is the effective neutron multiplication factor, $\Sigma_f$ 
  is the fission cross section, $\nu_f$ is the neutron yield per fission, and 
  $\Sigma_s(\vr,E' \rightarrow E)$ is the scattering cross section for neutrons 
  at position $\vr$ scattering from energy $E'$ to $E$.
  
  The neutron diffusion equation must be discretized in space and 
  energy to be solved computationally. Energy discretization is relatively 
  straight-forward.  An energy structure is described as $\{E_g\}$ for
  $g = 1,2,\ldots,G$ and by convention
  \[ E_G > E_{G-1} > \ldots > E_2 > E_1  \]
  Then, multigroup constants can be calculated based on the energy group 
  structure and the known cross sections. Multigroup constants are calculated to
  preserve the number of neutrons produced or destroyed. That is, the 
  calculation preserves reaction rates where the reaction rate for reaction $x$
  is defined as $R_x=\Sigma_x \phi$. Multigroup constants are then calculated. A
  formal derivation is given in \cite{duderstathamilton} and the results are 
  presented below.
  \begin{align}
    D_g(\vr) &= \frac{\int_{E_g}^{E_{g-1}} D(\vr,E') \grad \phi(\vr,E)\;dE}
      {\int_{E_g}^{E_{g-1}} \grad \phi(\vr,E)\;dE} \\
    \Sigma_{t,g}(\vr) &= \frac{\int_{E_g}^{E_{g-1}} \Sigma_t(\vr,E) 
      \phi(\vr,E)\;dE}{\int_{E_g}^{E_{g-1}} \phi(\vr,E)\;dE} \\
    \nu\Sigma_{f,g}(\vr) &= \frac{\int_{E_g}^{E_{g-1}} \nu_f(\vr,E)
      \Sigma_f(\vr,E) \phi(\vr,E)\;dE}{\int_{E_g}^{E_{g-1}} 
      \phi(\vr,E)\;dE} \\
    \Sigma_{s,g\rightarrow g'}(\vr) &= \frac{\int_{E_g'}^{E_{g'-1}} 
      \int_{E_g}^{E_{g-1}} \Sigma_s(\vr,E' \rightarrow E) \phi(\vr,E')\;dE\;dE'}
      {\int_{E_g}^{E_{g-1}} \phi(\vr,E)\;dE}  \\
    \chi_g(\vr) &= \int_{E_g}^{E_{g-1}} \chi(\vr,E) \; dE \\
    \phi_g(\vr) &= \int_{E_g}^{E_{g-1}} \phi(\vr,E) \; dE
  \end{align}
  Note that cross sections $\nu_f(\vr,E)$ and $\Sigma_f(\vr,E)$ have been 
  combined. This is necessary and a mathematical result of the group collapse.
  Then, \eref{eq:continuous_diffusion} can be discretized in energy as
  \begin{align}\label{eq:multigroup_diffusion}
    - \grad ( D_g(\vr) \grad \phi_g(\vr)) + \Sigma_{t,g}(\vr) \phi_g(\vr) = 
      \frac{\chi_g(\vr)}{\keff} \sum_{g'=1}^{G} \nu\Sigma_{f,g'}(\vr) 
      \phi_{g'}(\vr) + \sum_{g'=1}^{G} \Sigma_{s,g' \rightarrow g}(\vr) 
      \phi_{g'}(\vr)
  \end{align}
  Spatial dicretization will be based on the Finite Element Method (FEM) and 
  will be discussed in \sref{sec:formulation:derivation}.
  
  % todo discuss source splitting (up/down/self scatter) here or later
  The total neutron cross section includes the contribution due to 
  self-scattering. That is, due to $\Sigma_{s,g\rightarrow g}$. This can be 
  removed from \eref{eq:multigroup_diffusion} for simplicity.
  \begin{equation} \label{eq:multigroup_removal}
    - \grad ( D_g(\vr) \grad \phi_g(\vr)) + \Sigma_{r,g}(\vr) \phi_g(\vr) = 
      \frac{\chi_g(\vr)}{\keff} \sum_{g'=1}^{G} \nu\Sigma_{f,g'}(\vr) 
      \phi_{g'}(\vr) + \sum_{g'=1, g' \ne g}^{G} \Sigma_{s,g' \rightarrow g}(\vr) 
      \phi_{g'}(\vr)
  \end{equation}
  Where $\Sigma_{r,g}$ is the removal cross section and $\Sigma_{r,g}(\vr) = 
  \Sigma_{t,g}(\vr) - \Sigma_{s,g\rightarrow g}(\vr)$. For simplicity, the
  neutron sources in \eref{eq:multigroup_removal} can be  combined into a
  single term.
  \begin{equation} \label{eq:multigroup_source}
    - \grad ( D_g(\vr) \grad \phi_g(\vr)) + \Sigma_{r,g}(\vr) \phi_g(\vr) = 
      q_g(\vr)
  \end{equation}
  Where $q_g(\vr)$ is the combined neutron source at position $\vr$. $q$ can 
  then be further divided into contributions due to fission ($q_{fiss}$), up-
  scattering ($q_{up}$) when a neutron increases in energy, and down-scattering 
  when a neutron decreases in energy ($q_{down}$).
  \begin{align}
    q_g(\vr) &= q_{fiss,g}(\vr) + q_{up,g}(\vr) + q_{down,g}(\vr) \\
    q_{fiss,g}(\vr) &= \frac{\chi_g(\vr)}{\keff} \sum_{g'=1}^{G} 
      \nu \Sigma_{f,g'}(\vr) \phi_{g'}(\vr) \\
    q_{up,g}(\vr) &= \sum_{g'=g+1}^{G} \Sigma_{s,g' \rightarrow g}(\vr)
      \phi_{g'}(\vr) \\
    q_{down,g}(\vr) &= \sum_{g'=1}^{g} \Sigma_{s,g' \rightarrow g}(\vr)
      \phi_{g'}(\vr)
  \end{align}
  Where the difference between $q_{up}$ and $q_{down}$ are the limits of the 
  summation. In an iterative scheme, it will be necessary for fission and 
  up-scatter sources to use a different flux iterate than down-scatter so this
  division will prove useful
  

\section{Formulation}
  \subsection{Derivation}
    \label{sec:formulation:derivation}
    The only remaining continuous variable in the problem is the spatial 
    variable $\vr$. This will be discretized according to the Finite Element 
    Method (FEM). 
  \subsection{Matrix Quantities}
    \subsubsection{Linear Triangles}
    \subsubsection{Linear Wedges}

\section{Implementation}
  \subsection{Algorithm}
  	% in addition, make sure to mention the matrix is in RCM order
  \subsection{Memory and Storage}
  	% discuss two-table method
    % mention other alternatives
  \subsection{Linear System Solution}
  	% Dr. Kelly's book and CG Method
    % maybe discuss SPD here

\section{Reference Results}
	% develop a format to compare results for these so if updates happen
    % the csv or whatever can simply be updated
  \subsection{Triangular Element Manufactured Solutions}
    \subsubsection{One Dimension, One-Group, Fixed Source}
    \subsubsection{One Dimension, One-Group, Criticality}
    \subsubsection{One Dimension, Two-Group, Criticality}
    \subsubsection{One Dimension, One-Group, Two-Region, Criticality}
    \subsubsection{Two Dimension, One-Group, Criticality}
  \subsection{Two Dimension Reactors}
    \subsubsection{VVER440}
    \subsubsection{SNR}
    \subsubsection{HWR}
    \subsubsection{IAEA}
  \subsection{Wedge Element Manufactured Solution Finite Cylinder}
  \subsection{Three Dimension Reactors}
    \subsubsection{MONJU}
    \subsubsection{KNK}
