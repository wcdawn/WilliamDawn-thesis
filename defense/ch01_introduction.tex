\section{Introduction}
\label{sec:introduction}

\begin{frame}{Why are we here?}
  \begin{itemize}
    \item {\huge Model a nuclear reactor.}
      \pause
      \begin{itemize}
        \item Neutron distribution.
        \item Thermal Hydraulics.
        \item Thermal Expansion.
      \end{itemize}
  \end{itemize}
\end{frame}

\begin{frame}{But it's already been done!}
  \begin{itemize}
    \pause 
    \item You're right.
    \pause
    \item Computers are different now.
    \item Numerical methods are more efficient now.
    \item \textbf{FORTRAN} has \textit{a few} new standards.
  \end{itemize}
\end{frame}

\begin{frame}{Present Simulation Procedure}
  \begin{itemize}
    \item Heuristically estimate material temperatures.
    \item Manually calculate thermally expanded dimensions.
    \item Manually homogenization/smearing fractions and number densities.
    \item Run \dif and collect \keff and power distribution.
    \pause
  \end{itemize}
  \vspace{0.3in}
  \begin{block}{}
    \centering
    No thermal feedback or multiphysics simulation capability.\\
    Modern numerical methods can be implemented.
  \end{block}
\end{frame}

\begin{frame}{Goals}
  \begin{itemize}
    \item User input.
      \begin{itemize}
        \item Easy to use with intuitive keywords.
        \item Reactor geometry via VTK mesh.
        \item Temperature dependent cross sections either plain-text or ISOTXS
          format.
        \item Pin and assembly dimensions.
        \item Material compositions.
      \end{itemize}
    \item Simulate thermal expansion and thermal hydraulics internally.
    \item Collect \keff, reactor power, and average material temperatures.
  \end{itemize}
  \vspace{0.2in}
  \begin{itemize}
    \item Thermal expansion and thermal hydraulic simulations have been moved
      internally.
    \item Thermal hydraulic simulation improves the accuracy of neutronics
      simulation in the form of more accurate cross sections.
  \end{itemize}
\end{frame}
