\chapter{Discussion, Conclusions, and Recommendations}
\label{ch:conclusions}

\section{Discussion of Simulation Results}

  The purpose of this thesis is to simulate a fast nuclear power reactor at 
  operating conditions with coupled multiphysics simulations. The method
  developed allows for solution to the multigroup neutron diffusion equation for
  general unstructured mesh. The coupled multiphysics models allows for inherent
  modeling instead of thermal feedback effects instead of using extremely
  simplified and manual models. By including all simulations in a single
  simulation suite, a user can more easily observe the interaction of 
  physical phenomena and feedback.

  In \chref{ch:neutronDiffusion}, a rigorous and general framework is developed
  for solving the multigroup neutron diffusion equation using the \gls{fem} for
  general unstructured mesh. Insights are provided into the use of both
  two-dimension triangular elements and three-dimensional wedge (pentahedral)
  elements. Both of these geometries are natural choices for fast reactors which 
  typically employ hexagonal geometries. Using the developed methods,
  \chref{ch:diffusionResults} then demonstrates solution verification and
  solution validation for both analytic and reactor benchmark problems. The
  multigroup neutron diffusion solver as implemented is shown to converge to the
  correct answer at the correct convergence rate.

  \chref{ch:thermalHydraulics} presents the details of the thermal hydraulic
  models employed. These models include an axial convection model and a radial
  conduction model. The results of the thermal hydraulic calculations are
  material temperatures that are used to interpolate cross-section tables and
  update coolant density to generate temperature-dependent cross-sections.

  In \chref{ch:thermalExpansion}, a simplified thermal expansion model is
  presented. The model expands materials linearly assuming expansion as either
  HT9 Stainless-Steel structural material or U10Zr fuel material. The model
  requires an \textit{a priori} assumption of material temperatures but results
  are not highly sensitive to these temperatures due to the magnitudes of
  thermal expansion coefficients.

  Finally, \chref{ch:coupledResults} presents the culmination of all models
  implemented. A typical fast reactor as presented in a benchmark problem is 
  simulated.  Using the models developed, reactivity coefficients can be 
  estimated for an operating reactor. These coefficients describe dynamic 
  reactor behavior and agree with expected values. These reactivity coefficients 
  also describe the mechanisms for inherent safety in a fast reactor.

\section{Conclusions}
  
  It has been demonstrated that the \gls{fem} can be used to efficiently 
  simulate the power distribution in a nuclear power reactor. Use of the
  \gls{fem} has allowed for the local simulation of multiphysics effects within 
  elements. By simulating thermal hydraulic feedback and thermal expansion 
  effects, reactivity effects are estimated. Prior to this coupling method, the 
  simulation of feedback effects required either manual iterative process 
  between thermal hydraulic codes and neutron diffusion codes or the use of 
  simplified estimates of temperatures. 
  
\section{Recommendations for Future Research}

  The results of this thesis demonstrate a framework for an all-in-one reactor 
  simulator for fast reactor simulations. Future work includes code 
  enhancements, added features, and simulation of new reactors. Ultimately, the 
  goal is to develop a reactor simulation suite that can be used to perform core 
  design calculations and analyze dynamic reactive behavior.

  \subsection{Depletion Capabilities}
    To serve as a practical and useful reactor simulator, this suite must be
    expanded to include depletion capabilities. The code as presented in this
    work can be used to design a clean (i.e. undepleted) fast reactor. However,
    any reactor design must be capable of extended and continuous operation
    during an operating cycle. Depletion calculations inform the reactor design
    by establishing the required excess reactivity at the beginning of cycle.
    Control strategies must then be designed to control this excess reactivity. 

    Depletion calculations require solving the Bateman equations in the matrix
    exponential format. Preliminary investigations have been conducted for the
    implementation of the \gls{cram} \cite{cram}. The \gls{cram} will be
    implemented using the \gls{fem} mesh. A subsequent mesh study will be
    required to determine if it is possible to use a coarse mesh for refinement
    or if the fine mesh must also be used in depletion calculations.

  \subsection{Higher Order Finite Elements}
    Recall the \gls{fem} as presented in \chref{ch:neutronDiffusion} employs
    only linear triangular and wedge elements. The \gls{fem} as derived can be
    implemented for polynomials of arbitrary order. Work in \cite{Hosseini2013} 
    suggests that quadratic elements provide a significant reduction of error
    without the need to reduce the mesh spacing. With linear finite elements,
    the \gls{fem} is second-order spatially convergent. Quadratic finite
    elements convey third-order spatial convergence and cubic finite elements
    convey fourth-order spatial convergence. Incorporating higher order finite
    elements will allow for simple solution refinement without regenerating a
    mesh.

  \subsection{Encouraging Code Usage}
    The simulation suite is designed to be used for practical fast nuclear power
    reactor simulations. An important factor in this code being practical is its
    adoption by users. Encouraging the usage of the simulation package will
    allow for user feedback. Additionally, users employing the code in a variety
    of simulations will encourage the development of future features that may be
    necessary for unique reactor designs. Future possible reactors to be 
    simulated include the SuperPhenix benchmark and simulation of the
    \gls{ebr-ii} inherent safety demonstrations \cite{ebriitests}.
    
    \renewcommand{\thefootnote}{\fnsymbol{footnote}}
    Currently, the code is maintained in a private Github repository so it may 
    easily be shared in the future. The author may be contacted\footnotemark 
    for access to the repository. 

    \footnotetext{William C. Dawn: 
      \texttt{\href{mailto:wcdawn@ncsu.edu}{wcdawn@ncsu.edu}}}
