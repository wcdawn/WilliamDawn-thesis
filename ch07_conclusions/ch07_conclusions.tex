\chapter{Discussion, Conclusions, and Recommendations}
\label{ch:conclusions}

\section{Discussion of Simulation Results}

  The purpose of this thesis is to simulate a fast nuclear power reactor at 
  operating conditions with coupled multiphysics simulations. The method
  developed allows for solution to the multigroup neutron diffusion equation for
  general unstructured mesh. The coupled multiphysics models allows for inherent
  modeling instead of thermal feedback effects instead of using extremely
  simplified and manual models. By including all simulations in a single
  simulation suite, a user can more easily observe the interaction of 
  physical phenomena and feedback.

  In \chref{ch:neutronDiffusion}, a rigorous and general framework is developed
  for solving the multigroup neutron diffusion equation using the \gls{fem} for
  general unstructured mesh. Insights are provided into the use of both
  two-dimension triangular elements and three-dimensional wedge (pentahedral)
  elements. Both of these geometries are natural choices for fast reactors which 
  typically employ hexagonal geometries. Using the developed methods,
  \chref{ch:diffusionResults} then demonstrates solution verification and
  solution validation for both analytic and reactor benchmark problems. The
  multigroup neutron diffusion solver as implemented is shown to converge to the
  correct answer at the correct convergence rate.

  \chref{ch:thermalHydraulics} presents the details of the thermal hydraulic
  models employed. These models include an axial convection model and a radial
  conduction model. The results of the thermal hydraulic calculations are
  material temperatures that are used to interpolate cross-section tables and
  update coolant density to generate temperature-dependent cross-sections.

  In \chref{ch:thermalExpansion}, a simplified thermal expansion model is
  presented. The model expands materials linearly assuming expansion as either
  HT9 Stainless-Steel structural material or U10Zr fuel material. The model
  requires an \textit{a priori} assumption of material temperatures but results
  are not highly sensitive to these temperatures due to the magnitudes of
  thermal expansion coefficients.

  Finally, \chref{ch:coupledResults} presents the culmination of all models
  implemented. A typical fast reactor as presented in a benchmark problem is 
  simulated.  Using the models developed, reactivity coefficients can be 
  estimated for an operating reactor. These coefficients describe dynamic 
  reactor behavior and agree with expected values. These reactivity coefficients 
  also describe the mechanisms for inherent safety in a fast reactor.

\section{Conclusions}
  
  It has been demonstrated that the \gls{fem} can be used to efficiently 
  simulate the power distribution in a nuclear power reactor. Use of the
  \gls{fem} has allowed for the local simulation of multiphysics effects within 
  elements. By simulating thermal hydraulic feedback and thermal expansion 
  effects, reactivity effects are estimated. Prior to this coupling method, the 
  simulation of feedback effects required either manual iterative process 
  between thermal hydraulic codes and neutron diffusion codes or the use of 
  simplified estimates of temperatures. 
  
\section{Recommendations for Future Research}

  The results of this thesis demonstrate a framework for an all-in-one reactor 
  simulator for fast reactor simulations. Future work includes code 
  enhancements, added features, and simulation of new reactors. Ultimately, the 
  goal is to develop a reactor simulation suite that can be used to perform core 
  design calculations and analyze dynamic reactive behavior.

  \subsection{Encouraging Code Usage}
    

  \subsection{Code Enhancements}
    Depletion, Higher Order elements, \& SPN
 
  \subsection{Further Investigations}
    \begin{itemize}
      \item SuperPhenix Benchmark
      \item EBR-II modeling
    \end{itemize}
