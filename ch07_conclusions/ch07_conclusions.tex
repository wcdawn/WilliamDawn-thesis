\chapter{Summary, Conclusions, and Recommendations}
\label{ch:conclusions}

\section{Summary of Simulation Results}

  The purpose of this thesis is to simulate a fast nuclear power reactor at 
  operating conditions with coupled multiphysics feedback. The method
  developed allows for solution to the multigroup neutron diffusion equation for
  general unstructured mesh and an arbitrary energy group structure. The coupled 
  multiphysics models allows for inherent modeling thermal feedback effects 
  instead of using extremely simplified models or calculations by hand. By 
  including all multiphysics models in a single simulation suite, a user can 
  easily observe the interaction of physical phenomena and feedback.

  In \chref{ch:neutronDiffusion}, a rigorous and general framework is developed
  for solving the multigroup neutron diffusion equation via the \gls{fem} for
  unstructured mesh. Insights are provided into the use of both
  two-dimensional triangular elements and three-dimensional wedge (pentahedral)
  elements. Both of these geometries are natural choices for fast reactors which 
  typically employ hexagonal geometries. Using the methods developed,
  \chref{ch:diffusionResults} then demonstrates code verification and solution
  solution verification and for both analytic and reactor benchmark problems. 
  The multigroup neutron diffusion solver as implemented is shown to converge to 
  the correct $\keff$ value and flux distribution at the correct convergence 
  rate.

  \chref{ch:thermalHydraulics} presents the details of the thermal hydraulic
  models employed. These models include an axial heat convection model and a
  radial heat conduction model. Temperatures resulting from these thermal
  hydraulic models are used to interpolate cross section tables and update
  coolant density to generate temperature-dependent cross sections. The formulae
  for the interpolation of temperature-dependent cross sections are also
  provided.

  In \chref{ch:thermalExpansion}, a simplified thermal expansion model is
  presented. The model expands materials linearly assuming materials thermally
  expand as either HT9 Stainless-Steel structural material or U10Zr fuel 
  material. The model requires an \textit{a priori} assumption of material 
  temperatures but results are not highly sensitive to these temperatures due 
  to the magnitudes of thermal expansion coefficients.

  The culmination of these multiphysics models is presented in a coupled
  simulation in \chref{ch:coupledResults}. A typical fast reactor, presented in 
  a benchmark problem, is simulated \cite{abr}. Using the models developed, 
  reactivity coefficients can be estimated for an operating reactor. These 
  coefficients provide insights into dynamic reactor behavior and agree with 
  expected values. These reactivity coefficients also describe the mechanisms 
  for inherent safety in a fast reactor.

\section{Conclusions}
  
  It has been demonstrated that the \gls{fem} can be used to efficiently
  simulate the power distribution in a nuclear power reactor. Use of the
  \gls{fem} has allowed for the local simulation of multiphysics effects within
  elements.  The \gls{fem} framework allows for intuitive coupling of
  multiphysics effects because physics simulations can all share the same mesh.

  By simulating thermal hydraulic feedback and thermal expansion effects,
  reactivity effects are estimated. Prior to this work, the simulation of
  feedback effects required either manual iteration between thermal hydraulic
  calculations and a neutron diffusion solver or the use of other simplified
  estimates. This work has demonstrated a practical implementation of fast
  reactor multiphysics simulations by utilizing the \gls{fem} and a finite
  element framework.
  
\section{Recommendations for Future Research}

  In addition to general efficiency improvements (e.g. increased use of sparse
  matrices and vectorization) future work includes code enhancements, added 
  features, and simulation of new reactors. Ultimately, the goal is to develop a
  reactor simulation suite that can be used by industry professionals to perform 
  core design optimization calculations and analyze dynamic reactor behavior. 

  \subsection{Depletion Capabilities}
    To serve as a useful reactor simulator, this suite must be expanded to 
    include depletion capabilities. The code as presented in this work can be 
    used to design and optimize a clean (i.e. cycle zero) fast reactor. However,
    any practical reactor design must be capable of extended and continuous 
    operation during an operating cycle. Depletion calculations inform the 
    reactor design by establishing the required excess reactivity at the 
    beginning of cycle. Control strategies must then be designed to control 
    this excess reactivity. 

    Depletion calculations require solving the Bateman equations and the
    associated matrix exponential system of equations. Preliminary
    investigations have been conducted for the implementation of the \gls{cram}
    as the depletion solver in this suite \cite{cram}. The \gls{cram} will be 
    implemented using the \gls{fem} mesh. A subsequent mesh study will be
    required to determine if it is possible to use a coarse mesh for depletion
    or if the same fine mesh used for the multigroup neutron diffusion solution
    must also be used in depletion calculations.

  \subsection{Higher Order Finite Elements}
    Recall the \gls{fem} as presented in \chref{ch:neutronDiffusion} employs
    only linear triangular and linear wedge elements. However, the \gls{fem} as 
    derived can be implemented for polynomials of arbitrary order. Work by 
    \textcite{Hosseini2013} suggests that for finite element solutions to 
    the multigroup neutron diffusion equation, quadratic elements provide a 
    significant reduction of error without the need to reduce the mesh spacing. 
    With linear finite elements, the \gls{fem} is second-order spatially 
    convergent. Quadratic finite elements convey third-order spatial convergence 
    and cubic finite elements convey fourth-order spatial convergence. 
    Incorporating higher order finite elements will allow for simple solution 
    refinement without mesh regeneration.

  \subsection{\texorpdfstring{\glsentrylong{spn}}{Simplified PN} Solution}
    The \gls{spn} equations can provide considerably improved accuracy compared
    to the diffusion equation by incorporating angular dependence of flux rather
    than assuming isotropic scalar flux. The \gls{spn} equations reduce to a
    system of diffusion-like equations. Work by \textcite{Ryu2013} has
    demonstrated that the \gls{fem} can be used to solve the \gls{spn} equations 
    in general geometry. Implementation of the \gls{spn} equations would
    require the development of a new solver but much of the \gls{fem} framework
    can be reused. Implementation of the \gls{spn} method could provide enhanced
    accuracy for full-scale reactor problems.

  \subsection{Encouraging Code Usage}
    The simulation suite is designed to be used for practical fast nuclear power
    reactor simulations. An important step in this code becoming practical is
    its adoption by users. Encouraging the usage of the simulation package will
    allow for user feedback. Additionally, users employing the code in a variety
    of simulations will encourage the development of new features that may be
    necessary for unique reactor designs. Future possible reactors to be
    simulated include the SuperPhenix benchmark and simulation of the
    \gls{ebr-ii} inherent safety demonstrations \cite{ebriitests}.
    
    \renewcommand{\thefootnote}{\fnsymbol{footnote}}
    Currently, the code is maintained in a private Github repository so it may 
    easily be shared in the future. The author may be contacted\footnotemark 
    for access to the repository. 

    \footnotetext{William C. Dawn: 
      \texttt{\href{mailto:wcdawn@ncsu.edu}{wcdawn@ncsu.edu}}}


\glsresetall
