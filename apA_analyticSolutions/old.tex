\documentclass{article}
\usepackage{amsmath}
\usepackage{graphicx}
\usepackage{float}
\usepackage{hyperref}
\hypersetup{
  colorlinks,
  citecolor=black,
  filecolor=black,
  linkcolor=black,
  urlcolor=black
}
\begin{document}

\title{Manufactured Neutron Diffusion Problems}
\author{William C. Dawn}
\maketitle

\tableofcontents

\section{Introduction}
  The following is a manufactured solution designed for 1D and 2D Finite Element 
  Method (FEM) solvers in one and two spatial dimensions. Multigroup 
  problems are also addressed with a simplified one-dimensional problem.
  
  For the reference problems, the one-group neutron diffusion problem is written
  below.
  \begin{equation} \label{eq:onegroup}
    -D \nabla^2 \phi + \Sigma_r \phi =  \frac{1}{k_{eff}} \nu \Sigma_f \phi + 
      q_{fixed}
  \end{equation}
  For two-group neutron diffusion problems, the two-group neutron diffusion 
  problem is written below.
  \begin{align} 
    -D_1 \nabla^2 \phi_1 + \Sigma_{r1} \phi_1 &= \frac{1}{k_{eff}} \left(
      \nu \Sigma_{f1} \phi_1 + \nu \Sigma_{f2} \phi_2 \right) \\
    -D_2 \nabla^2 \phi_2 + \Sigma_{r2} \phi_2 &= 
      \Sigma_{s 1 \rightarrow 2} \phi_2
  \end{align}

Analytic solutions for 1D and 2D problems are provided below. These can be 
replicated in a 2D FEM solver using a square geometry. The reference square is 
$[0,1]\times[0,1]$. For one-dimension, two of the boundary conditions are set to
reflective conditions and two to essential $(\phi = 0)$ conditions. For 
two-dimension, all of the boundary conditions are set to essential conditions.

These formula are common but this specific formulation comes from E. E. Lewis
\textit{Fundamentals of Nuclear Reactor Physics}.

\section{One Dimension, One-Group, Fixed Source}
  For this problem, the following one-group coefficients are used.
  \begin{align*}
    D &= 1\\
    \Sigma_r &= 1\\
    \nu \Sigma_f &= 0\\
    q_{fixed} &= 1
  \end{align*}
  Then, equation \eqref{eq:onegroup} is written
  \begin{equation}
    - \nabla^2 \phi + \phi = 1 
  \end{equation}
  or
  \begin{equation} \label{eq:diffusion_simplified}
    \nabla^2 \phi - B^2 \phi = S
  \end{equation}
  where $B = 1$ and $S=-1$.
  The solution is composed of a particular and general solution.
  \begin{equation}
    \phi = \phi_g + \phi_p 
  \end{equation}
  For problems of the form of \eqref{eq:diffusion_simplified} the general 
  solution has the form
  \begin{equation} \label{eq:general}
    \phi_g = c_1 \cosh(Bx) + c_2 \sinh(Bx)
  \end{equation}
  And for constant $S=1$, the particular solution has the form
  \begin{align} \label{eq:particular}
    \phi_p &= -S/B \\
    \nonumber &= 1
  \end{align}
  Then, the solution has the form
  \begin{equation} 
    \phi = c_1 \cosh(Bx) + c_2 \sinh(Bx) + 1
  \end{equation}
  The only problem remaining is to solve for $c_1$ and $c_2$. These are given
  by the zero flux boundary conditions at the problem boundaries.
  \begin{align}
    \phi(0) &= 0\\
    \phi(L) &= 0
  \end{align}
  Evaluating the solution at 0, 
  \begin{align}
    \phi(0) &= 0 \\
    &= c_1 + 1\\
    c_1 &= -1
  \end{align}
  Evaluating the solution at $L$
  \begin{align}
    \phi(L) &= 0\\
    &= c_2 \sinh(BL) - \cosh(BL)-1\\
    c_2 &= \frac{\cosh(BL)-1}{\sinh(BL)}
  \end{align}
  The final solution is 
  \begin{equation} \label{eq:one_dimension}
    \phi(x) = -\cosh(Bx) + \frac{\cosh(BL)-1}{\sinh(BL)} \sinh(Bx) +1
  \end{equation}
  A plot is presented below
  \begin{figure}[H] 
    \centering
    \includegraphics[width=5in]{analytic_plot}
    \caption{Solution to 1D analytic problem.}
    \label{fg:analytic_plot}
  \end{figure}
  
\section{One Dimension, One Group, Criticality} \label{sc:onegroup1d}
  This problem also uses the one-group neutron diffusion equation from 
  \eqref{eq:onegroup} and the following coefficients:
  \begin{align*}
    D &= 1\\
    \Sigma_r &= 1\\
    \nu \Sigma_f &= 2\\
    q_{fixed} &= 0
  \end{align*}
  The problem is then proposed as 
  \begin{equation}
    -\nabla^2 \phi - \phi = 0 
  \end{equation}
  and has general solution
  \begin{equation}
    \phi_g = c_1 \cos(x) + c_2 \sin(x)
  \end{equation}
  and $c_1 = 0 $ for the given boundary conditions yielding
  \begin{equation} \label{eq:sinshape}
    \phi_g = c_2 \sin(x)
  \end{equation}
  The problem is an eigenvalue problem and has infinite solutions so the 
  constant $c_2$ is arbitrary and $\sin(x)=0$. Therefore, the geometric 
  bucking is $1/b$ where $b$ is the characteristic length of the problem. 
  Therefore, the solution is 
  \begin{equation} \label{eq:onedimensionsol}
    \phi = \sin(x/b)
  \end{equation}
  Plugging this solution back into \eqref{eq:onegroup} and dividing both sides
  by $\sin(x/b)$ will yield an expression for $k_{eff}$.
  \begin{align}
    -D B^2 + \Sigma_r &= \frac{1}{k_{eff}} \nu \Sigma_f \\
    k_{eff} &= \frac{\nu \Sigma_f}{DB^2 + \Sigma_r} \label{eq:keff1d}
  \end{align}
  Then, for a slab of length 100 cm, $B = \pi / 100$ and
  \[ k_{eff} = 1.9980280254 \]
\section{Two Dimension, One Group, Criticality}
  The same material coefficients are used for this problem as in the 
  one-dimensional problem. Similar to the one-dimensional problem, this 
  problem has basic solution of $\sin(x/b)$. The problem is separable in the 
  two spatial dimensions. 
  \begin{equation}
    \phi(x,y) = X(x) Y(y) 
  \end{equation}
  Beginning with equation \eqref{eq:onegroup}.
  \begin{align}
    -D \nabla^2 \phi(x,y) + \Sigma_r \phi(x,y) &= \nu \Sigma_f \phi(x,y) \\
    - \nabla^2 \phi(x,y) - \phi(x,y) &= 0 \\
    \nabla^2 \phi(x,y) + \phi(x,y) &= 0 \\
    \frac{\partial^2}{\partial x^2} \phi(x,y) + 
      \frac{\partial^2}{\partial y^2} \phi(x,y) +
      \phi(x,y) &= 0\\
    Y(y)\frac{\partial^2}{\partial x^2}X(x) +
      X(x) \frac{\partial^2}{\partial y^2} Y(y) + X(x)Y(y) &= 0\\
    Y(y)\left(\frac{\partial^2}{\partial x^2}X(x) + \frac{1}{2} X(x)\right)+
      X(x)\left(\frac{\partial^2}{\partial y^2}Y(y) + \frac{1}{2}Y(y)
      \right) &= 0
  \end{align}
  The solution to this problem is the product of the solutions from the 
  previous problem presented in \eqref{eq:onedimensionsol}. The solution is
  \begin{equation} \label{eq:twodimensionsol}
    \phi(x,y) = \sin(x/b) \sin(y/a)
  \end{equation}
  Where $b$ is the x-dimension of the problem and $a$ is the y-dimension of 
  the problem.
  If $B_x=B_y$; that is, $a=b$,  $k_{eff}$ can be calculated as
  \begin{equation}
    k_{eff} = \frac{\nu \Sigma_f}{2DB^2 + \Sigma_r} 
  \end{equation}
  Then, for $a = b = 100$
  \[ k_{eff} = 1.9960599356 \]
\section{One Dimension, Two Group, Criticality}
  Much of this section is drawn from a manuscript prepared by Dr. Scott 
  Palmtag \textit{Analytic Solution to 2-group Bare Core Problem}, June 2018.
  
  A two-group neutron diffusion problem is considered for a one-dimensional
  homogeneous slab. Energy group 1 is higher energy than energy group 2. 
  Regarding the energy structure, it is assumed all fission neutrons are 
  produced in the fast group (i.e. $\chi_1=1$ and $\chi_2=0$). Additionally, 
  it is assumed that there is no up-scattering 
  (i.e. $\Sigma_{2\rightarrow1}=0$).
  
  Then, the two-group neutron diffusion equations may be written
  \begin{align} \label{eq:twogroup}
  \begin{split}
    -D_1 \nabla^2 \phi_1 + \Sigma_{r1} \phi_1 &= \frac{1}{k_{eff}} \left(\nu 
      \Sigma_{f1} \phi_1 + \nu \Sigma_{f2} \phi_2 \right) \\
    -D_2 \nabla^2 \phi_2 + \Sigma_{r2} \phi_2 &= \Sigma_{s1\rightarrow2} 
      \phi_1 
  \end{split}
  \end{align}
  
  Referring to the problem posed by Section \ref{sc:onegroup1d}, the flux 
  solutions have form given in equation \eqref{eq:sinshape}. Then
  \begin{align} \label{eq:twogroupflux}
  \begin{split}
    \phi_1 &= c_1 \sin(Bx) \\
    \phi_2 &= c_2 \sin(Bx)
  \end{split}
  \end{align}
  Where $B$ is the geometric buckling term. For this geometry, the geometric 
  buckling of the fundamental mode is given by 
  \begin{equation}
    B = \pi/L
  \end{equation}
  where $L$ is the length of the problem. Then, plugging 
  \eqref{eq:twogroupflux} into the second equation from \eqref{eq:twogroup} 
  and dividing the equation by $\sin(Bx)$
  \begin{equation}
    D_2 B^2 c_2 + \Sigma_{r2} c_2 = \Sigma{s1\rightarrow2} c_1
  \end{equation}
  Then, the ratio can be expressed
  \begin{equation} \label{eq:fluxratio}
    \frac{c_2}{c_1} = \frac{\Sigma_{s1\rightarrow2}}{D_2 B^2 + \Sigma_{r2}}
  \end{equation}
  Returning to the first equation from \eqref{eq:twogroup} and again dividing
  both sides by $\sin(Bx)$
  \begin{align}
    D_1 B^2 c_1 &= \frac{1}{k_{eff}} \left( \nu \Sigma_{f1} c_1 + 
      \nu \Sigma_{f2} c_2\right)\\
    k_{eff} &= \frac{\nu \Sigma_{f1} c_1 + \nu \Sigma_{f2} c_2}
      {D_1 B^2 c_1 + \Sigma{r1} c_1}\\
    k_{eff} &= \frac{\nu \Sigma_{f1} + \nu \Sigma_{f2} c_2/c_1}
      {D_1 B^2 + \Sigma{r1}}
  \end{align}
  Plugging in the expression from \eqref{eq:fluxratio}
  \begin{equation}
    k_{eff} = \frac{\nu \Sigma_{f1} + \nu \Sigma_{f2} 
      \left(\frac{\Sigma_{s1\rightarrow2}}{D_2B^2+\Sigma_{r2}}\right)}
      {D_1 B^2 + \Sigma{r1}}
  \end{equation}
  From the VVER-440 benchmark, material 1,
  \begin{align*}
    D_1 &= 1.3466  \\
    D_2 &= 0.37169 \\
    \Sigma_{r1} &= 2.5255\text{E}-2\\
    \Sigma_{r2} &= 6.4277\text{E}-2\\
    \nu \Sigma_{f1}  &= 4.4488\text{E}-3\\
    \nu \Sigma_{f2}  &= 7.3753\text{E}-2\\
    \Sigma_{s1\rightarrow2} &= 1.6893\text{E}-2 \\
    k_{\infty} &= 0.943664259
  \end{align*}
  For a 100cm slab then,
  \begin{align*}
    \frac{c_2}{c_1} &= 0.26132419 \\
    k_{eff} &= 0.892349025
  \end{align*}
\section{One-Dimension, One-Group, Two-Region, ``Reactor''}
  This final analytic problem was designed to test the materials mapping of a
  FEM solver. The problem is proposed as a slab-reactor with a fuel and 
  reflector region. The neutron diffusion equation is given in 
  \eqref{eq:onegroup}. Geometry is described in the figure below.
  \begin{figure}[H] 
    \centering
    \includegraphics[width=3in]{2reg_geom}
    \caption{Geometry for Two-Region Problem.}
    \label{fg:2reg_geom}
  \end{figure}
  Boundary conditions on the horizontal surfaces (top/bottom) are reflective
  to reduce the problem to one-dimension. Other boundary conditions are given
  below.
  \begin{align}
    \phi'_F(0)&=0\\
    \phi_R(b)&=0\\
    D_F\phi'_F(a)&=D_R\phi'_R(a)
  \end{align}
  The material properties are specified below
  \begin{center}
  \begin{tabular}{c r r}
    & Fuel & Reflector\\
    $D$ & 1.2 & 0.7 \\
    $\Sigma_r$ & 0.02 & 0.015 \\
    $\nu \Sigma_f$ & 0.02 & 0 \\
    $q_{fixed}$ & 0 & 0
  \end{tabular}
  \end{center}
  The general form of the solution in the fueled region has the form 
  \[ \phi_F(x) = c_{1F} \cos(B_F x) + c_{2F} \sin(B_F x) \]
  Then, using the boundary condition $\phi'_F(0)=0$
  \begin{align}
    \phi_F(x) &= c_{1F} \cos(B_F x) + c_{2F} \sin(B_F x) \\
    \phi'_F(x) &= -B_F c_{1F} \sin(B_F x) + B_F c_{2F} \cos(B_F x) \\
    \phi'_F(0) &= 0\\
    &=B_F c_{2F}
  \end{align}
  Therefore, $c_{2F}=0$ and
  \begin{equation}
    \phi_F(x) = c_{1F} \cos(B_F x)
  \end{equation}
  Because this is an eigenvalue problem, $c_{1F}$ is arbitrary.
  
  Now for the solution in the reflector region. The solution has general form
  \begin{equation}
    \phi_R(x) = c_{1R} \cosh(B_R (x-a)) + c_{2R} \sinh(B_R (x-a))
  \end{equation}
  Treating the boundary condition $\phi_R(b)=0$
  \begin{align}
    \phi_R(x) &= c_{1R} \cosh(B_R (x-a)) + c_{2R} \sinh(B_R (x-a))\\
    \phi_R(b) &= 0 \\
    &= c_{1R} \cosh(B_R(b-a)) + c_{2R} \sinh(B_R(b-a))\\
    c_{1R} &= -c_{2R} \frac{\sinh(B_R(b-a))}{\cosh(B_R(b-a))}\\
    c_{1R} &= -c_{2R} \tanh(B_R(b-a))\\
    c_{2R} &= -c_{1R} \frac{1}{\tanh(B_R(b-a))} \label{eq:c2rnumber1}
  \end{align}
  Treating the current continuity boundary condition
  $D_F \phi'_F(a) = D_R \phi'_R(a)$.
  \begin{align}
    D_F \phi'_F(a) &= D_R \phi'_R(a) \\
    -D_F c_{1F} B_F \sin(B_F a) &= D_R B_R c_{2R} \\
    c_{2R} &= -\frac{D_F c_{1F} B_F \sin(B_F a)}{D_R B_R} \label{eq:c2rnumber2}
  \end{align}
  Treating the flux continuity boundary condition $\phi_F(a)=\phi_R(a)$.
  \begin{align}
    \phi_F(a)&=\phi_R(a) \\
    c_{1F} \cos(B_F a) &= c_{1R} \\
    c_{1R} &= c_{1F} \cos(B_F a) \label{eq:c1r}
  \end{align}
  Setting equations \eqref{eq:c2rnumber1} and \eqref{eq:c2rnumber2} equal.
  \begin{equation}
    - \frac{D_F c_{1F} B_F \sin(B_F a)}{D_R B_R} = -c_{1R} \tanh(B_R(b-a))
  \end{equation}
  Plugging in the the expression for $c_{1R}$ from \eqref{eq:c1r}
  \begin{align}
    - \frac{D_F c_{1F} B_F \sin(B_F a)}{D_R B_R} &=
      - \frac{c_{1F} \cos(B_F a)}{\tanh(B_R(b-a))}\\
    \frac{D_F B_F \sin(B_F a)}{D_R B_R} &= 
      \frac{\cos(B_F a)}{\tanh(B_R(b-a))} \\
    B_F \tan(B_F a) &= \frac{D_R B_R}{D_F \tanh(B_R(b-a))}
  \end{align}
  This is the most simplified solution. Unfortunately, there is no analytic
  expression for $B_F$. Therefore, a numeric solver must be used such as 
  MATLAB's \verb|vpasolve()| or a generic bisection/binary search. Because of
  the $\tan()$ function, the solution to $B_F$ is especially sensitive to the
  starting guess.
  
  Once $B_F$ is known, the problem is solved. By material definition, 
  ${B_R = \sqrt{\Sigma_{rR}/D_R}}$.
  \begin{align}
    \phi_F(x) &= \phi_0 \cos(B_F x)\\
    \phi_R(x) &= \phi_0 \cos(B_F a) \cosh(B_R (x-a)) 
      - \frac{D_F \phi_0 \sin(B_F a)}{D_R B_R} \sinh(B_R (x-a))\\
    \phi(x) &= H(x-a)\phi_R(x) + H(a-x)\phi_F(x)
  \end{align}
  Where $H(x)$ is the Heaviside step function.  
  
  The eigenvalue for this problem can be expressed for a known $B_F$. Similar 
  to \eqref{eq:keff1d} above, in this problem, for $a=50$, $b=100$
  \begin{align}
    B_F^2  &= 0.02559220482138791289 \\
    k_{eff} &= \frac{\nu\Sigma_{fF}}{D_F B_F^2 + \Sigma_{rF}} \\
    &= 0.96218825608052727105
  \end{align}
  
\section{One-Dimension, One-Group, Two-Region, Fixed-Source}
  This problem is similar to that described above except both regions are 
  non-fissile and there is a uniform fixed source in the problem. This will 
  draw on many previous equations.
  
  In this problem, material 1 is on the left and extends from 0 to $a$. 
  Material 2 is on the right and extends from $a$ to $b$. Boundary conditions 
  are $\phi(0)=\phi(b)=0$.
  
  The diffusion problem is written as \eqref{eq:onegroup} with 
  $\nu \Sigma_f =0$ and $q_{fixed}$ nonzero. Then, from \eqref{eq:particular},
  \begin{equation}
  	\phi_p = \frac{q_{fixed}}{\Sigma_r b}
  \end{equation}
  NOTE: This solution requires constant removal cross section in the two 
  materials. Otherwise, a different method for calculating the particular 
  solution must be used. The solution has general form similar to
  \eqref{eq:general}.
  \begin{align}
  	\phi &= \phi_p + \phi_1 H(a-x) + \phi_2 H(x-a) \\
    \phi_1 &= c_1 \cosh(B_1 x) + c_2 \sinh(B_1 x) \\
    \phi_2 &= c_3 \cosh(B_2 (x-a)) + c_4 \sinh(B_2 (x-a))
    \phi(0)&=\phi(b)=0
  \end{align}
  Where $B_i$ is the buckling coefficient for material $i$ given as
  \begin{equation}
    B_i = \sqrt{\Sigma_{ri} / D_i}
  \end{equation}
  
  Beginning with the boundary condition at $x=0$
  \begin{align}
    \phi(0) &= 0 \\
    &= \phi_p + c_1 \cosh(B_1 x) + c_2 \sinh(B_1 x) \\
    &= \phi_p + c_1\\
    c_1 &= -\phi_p \label{eq:bc_x0}
  \end{align}
  Next the boundary condition at $x=b$
  \begin{align}
    \phi(b) &= 0 \\
    &= \phi_p + c_3 \cosh(B_2 (b-a)) + c_4 \sinh(B_2 (b-a)) \\
    c_3 &= - \frac{\phi_p + c_4 \sinh(B_2(x-a))}{\cosh(B_2(b-a))} \\
    c_3 &= - \left(\frac{\phi_p}{\cosh(B_2(b-a))} + 
      c_4 \tanh(B_2(b-a))\right) \label{eq:bc_xb}
  \end{align}
  Next the interfacial boundary condition for conservation of flux
  $\phi(a^-)=\phi(a^+)$
  \begin{align}
    \phi(a^-) &= \phi(a^+) \\
    \phi_p - \phi_p \cosh(B_1 a) + c_2 \sinh(B_1 a) &= 
      \phi_p + c_3 \\
    c_3 &= c_2 \sinh(B_1 a) - \phi_p \cosh(B_1 a) \label{eq:bc_flux}
  \end{align}
  Next the interfacial boundary condition for conservation of current
  ${D_1 \phi'(a^-) = D_2 \phi'(a^+)}$
  \begin{align}
    D_1 \phi'(a^-) &= D_2 \phi'(a^+) \\
    D_1 (c_1 B_1 \sinh(B_1 a) + c_2 B_1 \cosh(B_1 a)) &=
      D_2 c_4 B_2 \\
    c_4 &= \frac{D_1 B_1}{D_2 B_2} (c_1 \sinh(B_1 a) + c_2 \cosh(B_1 a)) \\
    &= \frac{D_1 B_1}{D_2 B_2}(c_2 \cosh(B_1 a) - \phi_p \sinh(B_1 a))
      \label{eq:bc_current}
  \end{align}
  
  All of the coefficients have forward expressions given above except for 
  $c_2$. This will require significant algebra. Begin by setting 
  \eqref{eq:bc_xb} = \eqref{eq:bc_flux} and solve for $c_4$.
  \begin{align}
    - \left( \frac{\phi_p}{\cosh(B_2(b-a))} + c_4 \tanh(B_2 (b-a))\right)&=
      c_2 \sinh(B_1 a) - \phi_p \cosh(B_1 a) \\
    \phi_p \cosh(B_1 a) - \frac{\phi_p}{\cosh(B_2(b-a))} - 
      c_2 \sinh(B_1 a) &= c_4 \tanh(B_2 (b-a)) \\
    \phi_p \frac{\cosh(B_1 a)}{\tanh(B_2(b-a))} - 
      \frac{\phi_p}{\cosh(B_2(b-a))} 
      \frac{\cosh(B_2(b-a))}{\sinh(B_2(b-a))} - 
      c_2 \frac{\sinh(B_1 a)}{\tanh(B_2(b-a))} &= c_4\\
    \phi_p \left(\frac{\cosh(B_1 a)}{\tanh(B_2(b-a))} - \frac{1}
      {\sinh(B_2(b-a))} \right) - c_2 \frac{\sinh(B_1a)}{\tanh(B_2(b-a))} 
      &= c_4
  \end{align}
  Setting the above expression equal to \eqref{eq:bc_current} and finally 
  solve for $c_2$.
  \begin{align}
    \phi_p \left(\frac{\cosh(B_1 a)}{\tanh(B_2(b-a))} - 
      \frac{1}{\sinh(B_2(b-a))}\right) -
      c_2 \frac{\sinh(B_1 a)}{\tanh(B_2(b-a))} &=
      \frac{D_1 B_1}{D_2 B_2} \left( c_2 \cosh(B_1 a) - 
      \phi_1 \sinh(B_1 a)\right) \\
    \phi_p \left(\frac{\cosh(B_1 a)}{\tanh(B_2(b-a))} - 
      \frac{1}{\sinh(B_2(b-a))} \right) + 
      \frac{D_1 B_1}{D_2 B_2} \phi_p \sinh(B_1 a) &= 
      \frac{D_1 B_1}{D_2 B_2} c_2 \cosh(B_1a) + 
      c_2 \frac{\sinh(B_1a)}{\tanh(B_2(b-a))} \\
    \frac{\phi_p \left( \frac{\cosh(B_1a)}{\tanh(B_2(b-a))} - 
      \frac{1}{\sinh(B_2(b-a))} + 
      \frac{D_1 B_1}{D_2B_2} \sinh(B_1a) \right)}
      {\frac{D_1B_1}{D_2B_2} \cosh(B_1a) + 
      \frac{\sinh(B_1a)}{\tanh(B_2(b-a))}} &= c_2 \label{eq:c2}
  \end{align}
  Now, use equations \eqref{eq:bc_x0}, \eqref{eq:c2}, \eqref{eq:bc_flux}, and 
  \eqref{eq:bc_current} to solve for $c_1$ through $c_4$ respectively.
  
\section{Finite Cylinder, One Group, Criticality}
  The finite cylinder is an example of a truly three-dimensional problem with
  an analytic solution. This is a homogeneous cylinder with zero-flux boundary
  conditions on the edge of the cylinder. The cylinder has height $H$ and 
  radius $R$. The coordinates $r\in[0,R]$ and $z\in[0,H]$ where 
  $r=\sqrt{x^2+y^2}$.
  
  The solution method is by separation of variables into radial and axial 
  directions. Beginning with the axial direction. The diffusion equation can 
  be written
  \begin{equation} \label{eq:simplediffusion}
    \nabla^2 \phi(z) + B_z^2 \phi(z) = 0
  \end{equation}
  and has solution of the form
  \begin{equation} \label{eq:cyl_axial}
    \phi(z) = c_1 \cos(B_z z) + c_2 \sin(B_z z)
  \end{equation}
  Requiring $\phi(0)=\phi(H)=0$ yields $c_1=0$. Then $c_2$ is arbitrary and 
  the buckling condition is $B_zH=\pi$ and $B_z=\pi/H$.
  
  Moving on to the radial direction. The diffusion equation can be written.
  \begin{equation}
    \nabla^2 \phi(r) + B_r^2 \phi(r) = 0
  \end{equation}
  Noting the radial coordinates.
  \begin{align}
    \frac{1}{r} \frac{\partial}{\partial r} \left( r \frac{\partial \phi}
      {\partial r} \right) + B_r^2 \phi &= 0 \\
    \frac{\partial}{\partial r} \left( r \frac{\partial \phi}{\partial r}
      \right) + B_r^2 \phi &= 0
  \end{align}
  Noting the product rule of differentiation.
  \begin{align}
    r \frac{\partial^2 \phi}{\partial r^2} + \frac{\partial \phi}
      {\partial r} + B_r^2 r \phi &= 0 \\
    \frac{\partial^2 \phi}{\partial r^2} + \frac{1}{r} \frac{\partial \phi}
      {\partial r} + B_r^2 \phi &= 0 \label{eq:besselequation}
  \end{align}
  Noting equation \eqref{eq:besselequation}, Lewis Appendix B shows the 
  equation has solution of the form
  \begin{equation} \label{eq:cyl_radial}
    \phi(r) = c_3 J_0(B_r r) + c_4 Y_0(B_r r)
  \end{equation}
  Where $J_0$ is the Bessel function of the first kind, zeroth order and $Y_0$
  is the Bessel function of the second kind, zeroth order. Requiring the flux
  to be finite at $r=0$ requires $c_4=0$ as 
  $\lim_{r\rightarrow0} Y_0(r)=-\infty$. Again, $c_3$ is arbitrary. Zero flux 
  boundary conditions requires $B_r R=\alpha_0$ where $\alpha_0$ is the first 
  zero of the $J_0$ function and $\alpha_0 \approx 2.4048$. Then 
  $B_r=\alpha_0/R$.
  
  Combining the radial expression \eqref{eq:cyl_radial} and the axial 
  expression \eqref{eq:cyl_axial} yields the final expression for the flux.
  \begin{align} \label{eq:cyl_flux}
    \phi(r,z) &= J_0(B_r r) \sin(B_z z) \\
    &= J_0(r \alpha_0 / R) \sin(z \pi / H)
  \end{align}
  Plugging \eqref{eq:cyl_flux} into the diffusion equation will yield the 
  buckling/criticality condition.
  \begin{equation}
    -D \left( \frac{1}{r} \frac{\partial}{\partial r} \left( r 
      \frac{\partial \phi}{\partial r} \right) + \frac{\partial^2 \phi}
      {\partial z^2} \right) + \Sigma_r \phi = \frac{1}{k_{eff}} \nu 
      \Sigma_f \phi
  \end{equation}
  Beginning with the differentiation terms for simplicity. Axial 
  differentiation is straight-forward and presented below.
  \begin{equation}
    \frac{\partial^2 \phi}{\partial z^2} = -B_z^2 J_0(B_r r) \sin(B_z z)
  \end{equation}
  Radial differentiation must account for the radial geometry and is more 
  complex. First, note the following derivative relationship of the zeroth
  order Bessel function.
  \begin{equation} \label{eq:deriv_bessel0}
    \frac{d}{dr} J_0(\alpha r) = - \alpha J_1(\alpha r)
  \end{equation}
  \begin{align}
    \frac{\partial \phi}{\partial r} &= -B_r \sin(B_z z) J_1(B_r r) \\
    r \frac{\partial \phi}{\partial r} &= -B_r \sin(B_z z) r J_1 (B_r r) 
  \end{align}
  Note the additional derivative relation for the general Bessel function.
  \begin{equation} \label{eq:deriv_besseln}
    \frac{d}{dr} J_n(r) = \frac{1}{2} \left( J_{n-1}(r) - J_{n+1}(r)\right)
  \end{equation}
  Evaluating the rest of the cylindrical derivative using
  \eqref{eq:deriv_besseln} and the product rule.
  \begin{align}
    \frac{\partial}{\partial r} \left( r \frac{\partial \phi}{\partial r}
      \right) &= -B_r \sin(B_z z) \left(J_1(B_r r) + \frac{1}{2} B_r r \left(
      J_0(B_r r) - J_2(B_r r) \right) \right) \\
    \frac{1}{r} \frac{\partial}{\partial r} \left(r 
      \frac{\partial \phi}{\partial r} \right) &=
      -B_r \sin(B_z z) \left(\frac{1}{r} J_1(B_r r) + \frac{1}{2} B_r \left(
      J_0(B_r r) - J_2(B_r r) \right) \right)
  \end{align}
  Finally, plugging the expression for the Laplacian of the flux back into the
  diffusion equation.
  \begin{multline}
    D \left( B_r \sin(B_z z) \left( \frac{1}{r} J_1(B_r r) + \frac{1}{2} B_r
    \left( J_0(B_r r) - J_2(B_r r) \right) \right) + B_z^2 J_0(B_r r) \sin(B_z
    z) \right) + \\
    \Sigma_r J_0(B_r r) \sin(B_z z) \\
    = \frac{1}{k_{eff}} \nu
    \Sigma_f J_0(B_r r) \sin(B_z z)
  \end{multline}
  Dividing through by $\sin(B_z z)$.
  \begin{multline}
    D \left( B_r \left( \frac{1}{r} J_1(B_r r) + \frac{1}{2} B_r
    \left( J_0(B_r r) - J_2(B_r r) \right) \right) + B_z^2 J_0(B_r r) \right)+\\
    \Sigma_r J_0(B_r r) = \frac{1}{k_{eff}} \nu \Sigma_f J_0(B_r r) 
  \end{multline}
  Dividing through by $J_0(B_r r)$ and expanding some terms.
  \begin{align}
    D \left( B_r \left( \frac{1}{r} \frac{J_1(B_r r)}{J_0(B_r r)} + 
      \frac{1}{2} \left(B_r - \frac{J_2(B_r r)}{J_0(B_r r)} \right) \right) 
      + B_z^2 \right)+ \Sigma_r &= \frac{1}{k_{eff}} \nu \Sigma_f \\
    D \left( \frac{1}{r} B_r \frac{J_1(B_r r)}{J_0(B_r r)} + \frac{B_r^2}{2} -
      \frac{B_r^2}{2} \frac{J_2(B_r r)}{J_0(B_r r)} + B_z^2 \right) + \Sigma_r&=
      \frac{1}{k_{eff}} \nu \Sigma_f  \\
    \frac{1}{r} B_r \frac{J_1(B_r r)}{J_0(B_r r)} + \frac{B_r^2}{2} -
      \frac{B_r^2}{2} \frac{J_2(B_r r)}{J_0(B_r r)} + B_z^2 + 
      \frac{\Sigma_r}{D} &= \frac{1}{k_{eff}} \frac{\nu \Sigma_f}{D}\\
    \frac{1}{J_0(B_r r)} \frac{B_r^2}{2} \left(\frac{1}{r} \frac{2}{B_r} 
      J_1(B_r r) - J_2(B_r r) \right) + \frac{B_r^2}{2} + B_z^2 + 
      \frac{\Sigma_r}{D} &= \frac{1}{k_{eff}} \frac{\nu \Sigma_f}{D}
  \end{align}
  Note the Bessel function recursion relationship.
  \begin{equation} \label{eq:bessel_recursion}
    J_{n+1}(\alpha r) + J_{n-1}(\alpha r) = \frac{2n}{\alpha r} J_n(\alpha r)
  \end{equation}
  Using \eqref{eq:bessel_recursion} the term above is simplified.
  \begin{align}
    \frac{1}{J_0(B_r r)} \frac{B_r^2}{2} \left( J_0(B_r r) \right) + 
      \frac{B_r^2}{2} + B_z^2 + \Sigma_r &= \frac{1}{k_{eff}} 
      \frac{\nu \Sigma_f}{D} \\
    B_r^2 + B_z^2 + \frac{\Sigma_r}{D} &= \frac{1}{k_{eff}} \frac{\nu \Sigma_f}
      {D}
  \end{align}
  Now, an expression for the eigenvalue can be written.
  \begin{align}
    k_{eff} &= \frac{\nu \Sigma_f}{D(B_r^2 + B_z^2) + \Sigma_r} \\
    &= \frac{\nu \Sigma_f}{D\left(\left(\frac{\alpha_0}{R}\right)^2 + 
      \left(\frac{\pi}{H}\right)^2 \right) + \Sigma_r}
  \end{align}
  NOTE: the shape of the boundary is \textbf{extremely} important for this 
  problem. Therefore
  the mesh must be regenerated from scratch for each 
  refinement, rather than simply splitting nodes.
  
  
  
  
\end{document}
