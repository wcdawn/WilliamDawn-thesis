\chapter{Analytic Solutions to the Neutron Diffusion Equation}
\label{ap:analyticSolutions}

\section{Introduction}
  The following is a manufactured solution designed for verification of one-,
  two-, and three-dimension numerical neutron diffusion equation solvers.
  One-group and two-group problems are also addressed.
  
  For the reference problems, the one-group neutron diffusion problem is written
  below.
  \begin{equation} \label{eq:onegroup}
    -D \grad^2 \phi + \Sigma_r \phi =  \frac{1}{\keff} \nu \Sigma_f \phi + 
      q_{fixed}
  \end{equation}
  Where $\grad^2 \phi = \grad \cdot (\grad \phi)$ and is used to simplify 
  notation. For two-group neutron diffusion problems, the two-group neutron 
  diffusion problem is written below.
  \begin{align} 
    \label{eq:twogroup1}
    -D_1 \grad^2 \phi_1 + \Sigma_{r1} \phi_1 &= \frac{1}{\keff} \left(
      \nu \Sigma_{f1} \phi_1 + \nu \Sigma_{f2} \phi_2 \right) \\
    \label{eq:twogroup2}
    -D_2 \grad^2 \phi_2 + \Sigma_{r2} \phi_2 &= 
      \Sigma_{s 1 \rightarrow 2} \phi_2
  \end{align}
  Where $\phi_1$ is the higher energy group and $\phi_2$ is the lower energy 
  group. This formulation assumes all fission neutrons are created in the fast 
  energy group and there is no up-scattering (scattering that results in an 
  increase in neutron energy). These are realistic assumptions for all diffusive
  neutron systems.

  Analytic solutions are provided herein. One-dimension problems can be 
  replicated in a two-dimension solver using a square geometry and select 
  boundary conditions. For a given quadrilateral, two of the boundary conditions
  are set to reflective conditions and two are set to zero-flux $(\phi = 0)$ 
  conditions. For true two-dimensions problems, all of the boundary conditions 
  are set to zero-flux conditions.
  
  These formula are common to second order partial differential equations but
  the formulation here is based in part from \cite{textbooklewis}.

\section{One-Dimension, One-Group, Fixed Source}
  \label{sec:deriv_1dfixedsrc}
  This one-dimension problem is in the domain $x \in [0,L_x]$. The material is
  homogeneous within the problem and has fixed coefficient properties.
  The problem is specified in \eref{eq:1dfixed}, \eref{eq:1dfixed_bc1}, and 
  \eref{eq:1dfixed_bc2}.
  \begin{equation}
    \label{eq:1dfixed}
    -D \frac{d^2}{dx^2} \phi(x) + \Sigma_r \phi(x) = q_{fixed}
  \end{equation}
  \begin{align}
    \label{eq:1dfixed_bc1}
    \phi(0) &= 0 \\
    \label{eq:1dfixed_bc2}
    \phi(L_x) &= 0
  \end{align}
  Begin the solution by allowing to be composed of homogeneous and particular
  solutions. 
  \begin{equation} 
    \phi(x) = \phi_h(x) + \phi_p(x)
  \end{equation}
  The homogeneous solution satisfies equation \eref{eq:1dfixed_homog}.
  \begin{equation}
    \label{eq:1dfixed_homog}
    -D \grad^2 \phi_h(x) + \Sigma_r \phi_h(x) = 0
  \end{equation}
  The homogeneous solution $\phi_h(x)$ has form \eref{eq:1dfixed_homog_form}.
  \begin{equation}
    \label{eq:1dfixed_homog_form}
    \phi_h(x) = c_1 \, \cosh(\Sigma_r \, x) + c_2 \, \sinh(\Sigma_r \, x)
  \end{equation}
  The particular solution is given for a constant value $q_{fixed}$ in
  \eref{eq:1dfixed_particular}.
  \begin{equation}
    \label{eq:1dfixed_particular}
    \phi_p(x) = \frac{q_{fixed}}{\Sigma_r}
  \end{equation}
  Combining homogeneous and particular solutions.
  \begin{equation}
    \label{eq:1dfixed_constants}
    \phi(x) = c_1 \, \cosh(\Sigma_r x) + c_2 \, \sinh(\Sigma_r x) + 
      \frac{q_{fixed}}{\Sigma_r}
  \end{equation}
  Next, boundary conditions are considered with \eref{eq:1dfixed_constants}.
  Beginning with the boundary at $x=0$ as specified in \eref{eq:1dfixed_bc1}.
  \begin{align}
    \phi(0) &= 0 \\
    &= c_1 + \frac{q_{fixed}}{\Sigma_r} \\
    \label{eq:1dfixed_c1}
    c_1 &= - \frac{q_{fixed}}{\Sigma_r}
  \end{align}
  \eref{eq:1dfixed_c1} is then inserted into \eref{eq:1dfixed_constants}.
  \begin{align}
    \phi(x) &= - \frac{q_{fixed}}{\Sigma_r} \, \cosh(\Sigma_r x) + c_2 \,
      \sinh(\Sigma_r x) + \frac{q_fixed}{\Sigma_r} \\
    \phi(x) &= \frac{q_{fixed}}{\Sigma_r} \left( 1 - \cosh(\Sigma_r x) \right)
      + c_2 \sinh(\Sigma_r x)
  \end{align}




  
\section{One-Dimension, One-Group, Criticality} 
  \label{sec:deriv_1d1g}
  This one-dimension problem is in the domain $x \in [0,L]$.
  This problem also uses the one-group neutron diffusion equation from 
  \eref{eq:onegroup}. For this problem, and the following coefficients:
  \begin{align*}
    D &= 1\\
    \Sigma_r &= 1\\
    \nu \Sigma_f &= 2\\
    q_{fixed} &= 0
  \end{align*}
  The problem is then proposed as 
  \begin{equation}
    -\grad^2 \phi(x) - \phi(x) = 0 
  \end{equation}
  and has general solution
  \begin{equation} \label{eq:critical_general}
    \phi_g(x) = c_1 \cos(x) + c_2 \sin(x)
  \end{equation}
  Considering zero-flux boundary conditions from \eref{eq:bcx0} and 
  \eref{eq:bcxL}, then $c_1 = 0 $ yielding
  \begin{equation} \label{eq:sinshape}
    \phi_g(x) = c_2 \sin(x)
  \end{equation}
  The problem is an eigenvalue problem and has infinite solutions so the 
  constant $c_2$ is arbitrary for $\sin(x)=0$. Introducing a Buckling term $B$
  then $B=\frac{n \pi}{L}$ and $\sin(Bx)=0$ for $n$ an odd integer. The 
  fundamental mode is then $B=\frac{1}{L}$. Therefore, the solution is 
  \begin{equation} \label{eq:analytic_1d1g}
    \phi(x) = \sin(x/L)
  \end{equation}
  Inserting this solution back into \eref{eq:onegroup} and dividing both sides
  by $\sin(x/L)$ will yield an expression for $\keff$.
  \begin{align}
    -D B^2 + \Sigma_r &= \frac{1}{\keff} \nu \Sigma_f \\
    \keff &= \frac{\nu \Sigma_f}{DB^2 + \Sigma_r} \label{eq:keff1d}
  \end{align}
  Then, for a slab of length $ L = 100 \units{cm} $, $B = \pi / 100$ and
  \begin{equation}
    \keff = 1.9980280254
  \end{equation}
  
\section{Two-Dimension, One-Group, Criticality}
  \label{sec:deriv_2d1g}
  The same material coefficients are used for this problem as in the 
  one-dimension, criticality problem. For this problem, the domain is in the 
  rectangle $[0,L_x]\times[0,L_y]$. Similar to the one-dimension problem, 
  this problem has basic solution of $\sin(x/L)$. The problem is separable in 
  the two spatial dimensions. 
  \begin{equation}
    \phi(x,y) = X(x) Y(y) 
  \end{equation}
  Beginning with equation \eref{eq:onegroup}.
  \begin{align}
    -D \grad^2 \phi(x,y) + \Sigma_r \phi(x,y) &= \nu \Sigma_f \phi(x,y) \\
    - \grad^2 \phi(x,y) - \phi(x,y) &= 0 \\
    \grad^2 \phi(x,y) + \phi(x,y) &= 0 \\
    \frac{\partial^2}{\partial x^2} \phi(x,y) + 
      \frac{\partial^2}{\partial y^2} \phi(x,y) +
      \phi(x,y) &= 0\\
    Y(y)\frac{\partial^2}{\partial x^2}X(x) +
      X(x) \frac{\partial^2}{\partial y^2} Y(y) + X(x)Y(y) &= 0\\
    Y(y)\left(\frac{\partial^2}{\partial x^2}X(x) + \frac{1}{2} X(x)\right)+
      X(x)\left(\frac{\partial^2}{\partial y^2}Y(y) + \frac{1}{2}Y(y)
      \right) &= 0
  \end{align}
  For non-trivial solutions $X(x) \ne 0$ and $Y(y) \ne 0$ then the following
  second order ordinary differential equations can be solved.
  \begin{align}
    \frac{d^2}{dx^2} X(x) + \frac{1}{2} X(x) &= 0 \\
    \frac{d^2}{dy^2} Y(y) + \frac{1}{2} Y(y) &= 0 \\
  \end{align}
  The solution to these problems is the product of the solutions from the 
  previous problem presented in \eref{eq:analytic_1d1g}. The solution is
  \begin{equation} \label{eq:analytic_2d1g}
    \phi(x,y) = \sin(x/A) \sin(y/B)
  \end{equation}
  Where $L_x$ is the horizontal extent of the rectangular domain and $L_y$ is 
  the vertical extent of the rectangular domain. Extending the concept of 
  geometric buckling to the two spatial dimensions where $B_x = \frac{1}{L_x}$
  and $B_y = \frac{1}{L_y}$. If $B=B_x=B_y$; that is, $L_x=L_y$, $\keff$ can 
  be calculated as
  \begin{equation}
    \keff = \frac{\nu \Sigma_f}{2DB^2 + \Sigma_r} 
  \end{equation}
  Then, for $A=B = 100 \units{cm}$
  \begin{equation}
    \keff = 1.9960599356
  \end{equation}
  
\section{One-Dimension, Two-Group, Criticality}
  \label{sec:deriv_1d2g}
  Much of this section is drawn from course notes prepared by Dr. Scott 
  Palmtag \cite{analytic2g}.
  
  A two-group neutron diffusion problem is considered for a one-dimension
  homogeneous slab in the domain $x \in [0,L]$. Energy group 1 is higher energy
  than energy group 2. It is assumed all fission neutrons are  produced in the
  fast group (i.e. $\chi_1=1$ and $\chi_2=0$). Additionally, it is assumed that 
  there is no up-scattering  (i.e. $\Sigma_{2\rightarrow1}=0$).  Then, the
  two-group neutron diffusion equations are given in \eref{eq:twogroup1} and 
  \eref{eq:twogroup2}.
  
  Referring to the problem posed by Section \sref{sec:deriv_1d1g}, the flux 
  solutions have form given in equation \eref{eq:sinshape}. Then
  \begin{align}
    \label{eq:twogroupflux1}
    \phi_1(x) &= c_1 \sin(Bx) \\
    \label{eq:twogroupflux2}
    \phi_2(x) &= c_2 \sin(Bx)
  \end{align}
  Where $B$ is the geometric buckling term. For this geometry, the geometric 
  buckling of the fundamental mode is given by 
  \begin{equation}
    B = \pi/L_x
  \end{equation}
  Then, inserting \eref{eq:twogroupflux1} and \eref{eq:twogroupflux2} into 
  \eref{eq:twogroup2} and dividing the equation by $\sin(Bx)$
  \begin{equation}
    D_2 B^2 c_2 + \Sigma_{r2} c_2 = \Sigma{s1\rightarrow2} c_1
  \end{equation}
  Then, the ratio can be expressed
  \begin{equation} \label{eq:fluxratio}
    \frac{c_2}{c_1} = \frac{\Sigma_{s1\rightarrow2}}{D_2 B^2 + \Sigma_{r2}}
  \end{equation}
  The ratio $\frac{c_2}{c_1}$ represents the relative magnitude of thermal to 
  fast flux. Returning to \eref{eq:twogroup1}, and using in 
  \eref{eq:twogroupflux1} and \eref{eq:twogroupflux2} and again dividing both
  sides by  $\sin(Bx)$
  \begin{align}
    D_1 B^2 c_1 &= \frac{1}{\keff} \left( \nu \Sigma_{f1} c_1 + 
      \nu \Sigma_{f2} c_2\right)\\
    \keff &= \frac{\nu \Sigma_{f1} c_1 + \nu \Sigma_{f2} c_2}
      {D_1 B^2 c_1 + \Sigma{r1} c_1}\\
    \keff &= \frac{\nu \Sigma_{f1} + \nu \Sigma_{f2} c_2/c_1}
      {D_1 B^2 + \Sigma{r1}}
  \end{align}
  Inserting the expression from \eref{eq:fluxratio}
  \begin{equation}
    \keff = \frac{\nu \Sigma_{f1} + \nu \Sigma_{f2} 
      \left(\frac{\Sigma_{s1\rightarrow2}}{D_2B^2+\Sigma_{r2}}\right)}
      {D_1 B^2 + \Sigma{r1}}
  \end{equation}
  From the VVER-440 benchmark \sref{sec:vver440}, material 1,
  \begin{align*}
    D_1 &= 1.3466  \\
    D_2 &= 0.37169 \\
    \Sigma_{r1} &= 2.5255\text{E}-2\\
    \Sigma_{r2} &= 6.4277\text{E}-2\\
    \nu \Sigma_{f1}  &= 4.4488\text{E}-3\\
    \nu \Sigma_{f2}  &= 7.3753\text{E}-2\\
    \Sigma_{s1\rightarrow2} &= 1.6893\text{E}-2 \\
    k_{\infty} &= 0.943664259
  \end{align*}
  For a 100 \units{cm} slab.
  \begin{align}
    \frac{c_2}{c_1} &= 0.26132419 \\
    \keff &= 0.892349025
  \end{align}
  
\section{One-Dimension, One-Group, Two Region, Criticality}
  \label{sec:deriv_2reg}
  This final analytic problem was designed to test the materials mapping of a
  solver. The problem is proposed as a slab reactor with a fuel and reflector
  region. The neutron diffusion equation is given in  \eref{eq:onegroup}. 
  Geometry is described in \fref{fig:2reg_geom}. Fuel Material (subscript F) 
  is located in $x \in [0,a]$ and Reflector Material (subscript R) is located in
  $x \in [a,b]$.
  \begin{figure}
    \centering
    \includegraphics[width=0.6\textwidth]{2reg_geom}
    \caption{Geometry for Two Region Problem.}
    \label{fig:2reg_geom}
  \end{figure}
  Boundary conditions on the horizontal surfaces (top/bottom) are reflective
  to reduce the problem to one-dimension. Other boundary conditions are given
  below. Noting $\phi'(x) = \frac{d \phi}{dx}$.
  \begin{align}
    \label{eq:2reg_mirror}
    \phi'_F(0)&=0\\
    \label{eq:2reg_continuity}
    D_F\phi'_F(a)&=D_R\phi'_R(a) \\
    \label{eq:2reg_fluxcontinuity}
    \phi_F(a) &= \phi_R(a) \\
    \label{eq:2reg_zeroflux}
    \phi_R(b)&=0
  \end{align}
  These represent mirror condition at $x=0$, current continuity at $x=a$, flux
  continuity at $x=a$, and zero-flux at $x=b$. The material properties are 
  specified in \tref{tab:2reg_constants}.
  \begin{table}
    \caption{Two Region Material Constants.}
    \label{tab:2reg_constants}
    \begin{center}
      \begin{tabular}{crr}
        \toprule
        & Fuel & Reflector \\
        \midrule
        $D$ & 1.2 & 0.7 \\
        $\Sigma_r$ & 0.02 & 0.015 \\
        $\nu \Sigma_f$ & 0.02 & 0 \\
        $q_{fixed}$ & 0 & 0 \\
        \bottomrule
      \end{tabular}
    \end{center}
  \end{table}
  
  The general form of the solution in the fueled region has form similar to 
  \eref{eq:critical_general}.
  \begin{equation}
    \phi_F(x) = c_{1F} \cos(B_F x) + c_{2F} \sin(B_F x)
  \end{equation}
  Then, using the mirror boundary condition at $x=0$ \eref{eq:2reg_mirror}.
  \begin{align}
    \phi_F(x) &= c_{1F} \cos(B_F x) + c_{2F} \sin(B_F x) \\
    \phi'_F(x) &= -B_F c_{1F} \sin(B_F x) + B_F c_{2F} \cos(B_F x) \\
    \phi'_F(0) &= 0\\
    &=B_F c_{2F}
  \end{align}
  Therefore, $c_{2F}=0$ and
  \begin{equation}
    \phi_F(x) = c_{1F} \cos(B_F x)
  \end{equation}
  Because this is an eigenvalue problem, $c_{1F}$ is arbitrary and $B_F$ 
  represents the buckling factor in the Fuel Material.
  
  Now for the solution in the reflector region. The solution has general form
  similar to \eref{eq:fixedsource_general}.
  \begin{equation}
    \phi_R(x) = c_{1R} \cosh(B_R (x-a)) + c_{2R} \sinh(B_R (x-a))
  \end{equation}
  The coordinate transform $(x-a)$ is equivalent to multiplying by a constant 
  because $\sinh$ and $\cosh$ can be rewritten as exponentials and due to 
  exponential properties. Treating the zero-flux boundary condition at $x=b$ 
  \eref{eq:2reg_zeroflux}.
  \begin{align}
    \phi_R(x) &= c_{1R} \cosh(B_R (x-a)) + c_{2R} \sinh(B_R (x-a))\\
    \phi_R(b) &= 0 \\
    &= c_{1R} \cosh(B_R(b-a)) + c_{2R} \sinh(B_R(b-a))\\
    c_{1R} &= -c_{2R} \frac{\sinh(B_R(b-a))}{\cosh(B_R(b-a))}\\
    c_{1R} &= -c_{2R} \tanh(B_R(b-a))\\
    c_{2R} &= -c_{1R} \frac{1}{\tanh(B_R(b-a))} \label{eq:c2rnumber1}
  \end{align}
  Treating the current continuity boundary condition at $x=a$ 
  \eref{eq:2reg_continuity}.
  \begin{align}
    D_F \phi'_F(a) &= D_R \phi'_R(a) \\
    -D_F c_{1F} B_F \sin(B_F a) &= D_R B_R c_{2R} \\
    c_{2R} &= -\frac{D_F c_{1F} B_F \sin(B_F a)}{D_R B_R} \label{eq:c2rnumber2}
  \end{align}
  Treating the flux continuity boundary condition at $x=a$ 
  \eref{eq:2reg_fluxcontinuity}.
  \begin{align}
    \phi_F(a)&=\phi_R(a) \\
    c_{1F} \cos(B_F a) &= c_{1R} \\
    c_{1R} &= c_{1F} \cos(B_F a) \label{eq:c1r}
  \end{align}
  Setting equations \eref{eq:c2rnumber1} and \eref{eq:c2rnumber2} equal.
  \begin{equation}
    - \frac{D_F c_{1F} B_F \sin(B_F a)}{D_R B_R} = -c_{1R} \tanh(B_R(b-a))
  \end{equation}
  Using the the expression for $c_{1R}$ from \eref{eq:c1r}
  \begin{align}
    - \frac{D_F c_{1F} B_F \sin(B_F a)}{D_R B_R} &=
      - \frac{c_{1F} \cos(B_F a)}{\tanh(B_R(b-a))}\\
    \frac{D_F B_F \sin(B_F a)}{D_R B_R} &= 
      \frac{\cos(B_F a)}{\tanh(B_R(b-a))} \\
    B_F \tan(B_F a) &= \frac{D_R B_R}{D_F \tanh(B_R(b-a))}
  \end{align}
  By material definition,  ${B_R = \sqrt{\Sigma_{rR}/D_R}}$.
  
  This is the most simplified solution. Unfortunately, there is no analytic
  expression for $B_F$. Therefore, a numeric solver must be used such as 
  MATLAB's \texttt{vpasolve()} or a generic bisection/binary search. The $\tan()$ 
  function causes the solution to $B_F$ is especially sensitive to the starting
  guess.
  
  Once $B_F$ is known, the problem is solved.
  \begin{align}
    \phi_F(x) &= \phi_0 \cos(B_F x)\\
    \phi_R(x) &= \phi_0 \cos(B_F a) \cosh(B_R (x-a)) 
      - \frac{D_F \phi_0 \sin(B_F a)}{D_R B_R} \sinh(B_R (x-a))\\
    \label{eq:analytic_2reg}
    \phi(x) &= H(x-a)\phi_R(x) + H(a-x)\phi_F(x)
  \end{align}
  Where $H(x)$ is the Heaviside step function.  
  
  The eigenvalue for this problem can be expressed for a known $B_F$. Similar 
  to \eref{eq:keff1d}. In this problem, for $a=50 \units{cm}$ and
  $b=100 \units{cm}$ and material constants given in \tref{tab:2reg_constants}.
  \begin{align}
    B_F &= 0.0255922048213879 \\
    \keff &= \frac{\nu\Sigma_{fF}}{D_F B_F^2 + \Sigma_{rF}} \\
    &= 0.9621882561
  \end{align}
\section{Finite-Cylinder, One-Group, Criticality}
  \label{sec:deriv_finite_cyl}
  The finite cylinder is an example of a truly three-dimension problem with
  an analytic solution. This is a homogeneous cylinder with zero-flux boundary
  conditions on the edge of the cylinder. The cylinder has height $H$ and 
  radius $T$. The coordinates $r\in[0,T]$ and $z\in[0,H]$ where 
  $r=\sqrt{x^2+y^2}$.
  
  The solution method is by separation of variables into radial and axial 
  directions. This derivation is similar to that in \sref{sec:deriv_2d1g}.
  Beginning with the one-group neutron diffusion equation \eref{eq:onegroup} 
  and assuming that  $\phi(r,z) = R(r) Z(z)$.
  \begin{align}
    -D \grad^2 \phi(r,z) + \Sigma_r \phi(r,z) &= \nu \Sigma_f \phi(r,z) \\
    \grad^2 \phi(r,z) + \frac{\nu\Sigma_f - \Sigma_r}{D} \phi(r,g) &= 0 \\
    \frac{1}{r} \frac{\partial}{\partial r} \left( r \frac{\partial}{\partial r}
      \phi(r,z) \right) + \frac{\partial^2}{\partial z^2} \phi(r,z) +
      \frac{\nu\Sigma_f - \Sigma_r}{D} \phi(r,z) &= 0 \\
    Z(z) \frac{1}{r} \frac{\partial}{\partial r} \left( r 
      \frac{\partial}{\partial r} R(r) \right) + 
      R(r) \frac{\partial^2}{\partial z^2} Z(z) + 
      \frac{\nu\Sigma_f - \Sigma_r}{D} R(r) Z(z) &= 0 \\
    Z(z) \frac{1}{r} \frac{\partial}{\partial r} \left( r 
      \frac{\partial}{\partial r} R(r) \right) + 
      R(r) \frac{\partial^2}{\partial z^2} Z(z) + 
      B^2 R(r) Z(z) &= 0 \\
    Z(z) \left( \frac{1}{r} \frac{\partial}{\partial r} \left( r 
      \frac{\partial}{\partial r} R(r) \right) + \frac{B^2}{2} R(r) \right) + 
      R(r) \left( \frac{\partial^2}{\partial z^2} Z(z) + \frac{B^2}{2} Z(z) 
      \right) &= 0
  \end{align}
  For non-trivial solutions $R(r) \ne 0$ and $Z(z) \ne 0$ then the following
  second order ordinary differential equations can be solved.
  \begin{align}
    \label{eq:cyl_radialR}
    \frac{1}{r} \frac{\partial}{\partial r} \left( r \frac{\partial}{\partial r}
      R(r) \right) + \frac{B^2}{2} R(r) &= 0 \\
    \label{eq:cyl_axialZ}
    \frac{\partial^2}{\partial z^2} Z(z) + \frac{B^2}{2} Z(z) &= 0
  \end{align}
  
  Beginning with the axial direction. The diffusion equation can be rewritten as 
  \eref{eq:cyl_axialZ}.
  \begin{equation} \label{eq:simplediffusion}
    \grad^2 Z(z) + B_z^2 Z(z) = 0
  \end{equation}
  and has solution of the form \eref{eq:critical_general}.
  \begin{equation} \label{eq:cyl_axial}
    Z(z) = c_1 \cos(B_z z) + c_2 \sin(B_z z)
  \end{equation}
  Requiring zero-flux boundary conditions $\phi(0)=\phi(H)=0$ yields $c_1=0$. 
  Then $c_2$ is arbitrary and the buckling condition is $B_zH=\pi$ and $B_z=\pi/H$.
  
  Moving on to the radial direction. The diffusion equation can be written as
  \eref{eq:cyl_radialR}.
  \begin{equation}
    \grad^2 R(r) + B_r^2 R(r) = 0
  \end{equation}
  Noting the radial coordinates.
  \begin{align}
    \frac{1}{r} \frac{\partial}{\partial r} \left( r \frac{\partial R}
      {\partial r} \right) + B_r^2 R &= 0 \\
    \frac{\partial}{\partial r} \left( r \frac{\partial R}{\partial r}
      \right) + B_r^2 R &= 0
  \end{align}
  Noting the product rule of differentiation.
  \begin{align}
    r \frac{\partial^2 R}{\partial r^2} + \frac{\partial R}
      {\partial r} + B_r^2 r R &= 0 \\
    \frac{\partial^2 \phi}{\partial r^2} + \frac{1}{r} \frac{\partial R}
      {\partial r} + B_r^2 R &= 0 \label{eq:besselequation}
  \end{align}
  Noting equation \eref{eq:besselequation}, \cite{textbooklewis} Appendix B
  shows the equation has solution of the form
  \begin{equation} \label{eq:cyl_radial}
    R(r) = c_3 J_0(B_r r) + c_4 Y_0(B_r r)
  \end{equation}
  Where $J_0$ is the Bessel function of the first kind, zeroth order and $Y_0$
  is the Bessel function of the second kind, zeroth order. Requiring the flux
  to be finite at $r=0$ requires $c_4=0$ as 
  $\lim_{r\rightarrow0} Y_0(r) \rightarrow -\infty$. As this is an eigenvalue
  problem, $c_3$ is arbitrary. Zero flux boundary conditions requires 
  $B_r R=\alpha_0$ where $\alpha_0$ is the first root of the $J_0$ function and 
  $\alpha_0 \approx 2.4048$. Then $B_r=\alpha_0/R$.
  
  Combining the radial expression \eref{eq:cyl_radial} and the axial 
  expression \eref{eq:cyl_axial} yields the final expression for the flux.
  \begin{align} \label{eq:analytic_finite_cyl}
    \phi(r,z) &= J_0(B_r r) \sin(B_z z) \\
    &= J_0(r \alpha_0 / R) \sin(z \pi / H)
  \end{align}
  Inserting \eref{eq:analytic_finite_cyl} into the diffusion equation will yield 
  the buckling/criticality condition.
  \begin{equation}
    -D \left( \frac{1}{r} \frac{\partial}{\partial r} \left( r 
      \frac{\partial \phi}{\partial r} \right) + \frac{\partial^2 \phi}
      {\partial z^2} \right) + \Sigma_r \phi = \frac{1}{\keff} \nu 
      \Sigma_f \phi
  \end{equation}
  Beginning with the differentiation terms for simplicity. Axial 
  differentiation is straight-forward and presented below.
  \begin{equation}
    \label{eq:above}
    \frac{\partial^2 \phi}{\partial z^2} = -B_z^2 J_0(B_r r) \sin(B_z z)
  \end{equation}
  Radial differentiation must account for the radial geometry and is more 
  complex. First, note the following derivative relationship of the zeroth
  order Bessel function.
  \begin{equation} \label{eq:deriv_bessel0}
    \frac{d}{dr} J_0(\alpha r) = - \alpha J_1(\alpha r)
  \end{equation}
  And applying \eref{eq:deriv_bessel0} to \eref{eq:above}
  \begin{align}
    \frac{\partial \phi}{\partial r} &= -B_r \sin(B_z z) J_1(B_r r) \\
    r \frac{\partial \phi}{\partial r} &= -B_r \sin(B_z z) r J_1 (B_r r) 
  \end{align}
  Note the additional derivative relation for the general Bessel function.
  \begin{equation} \label{eq:deriv_besseln}
    \frac{d}{dr} J_n(r) = \frac{1}{2} \left( J_{n-1}(r) - J_{n+1}(r)\right)
  \end{equation}
  Evaluating the rest of the cylindrical derivative using
  \eref{eq:deriv_besseln} and the product rule.
  \begin{align}
    \frac{\partial}{\partial r} \left( r \frac{\partial \phi}{\partial r}
      \right) &= -B_r \sin(B_z z) \left(J_1(B_r r) + \frac{1}{2} B_r r \left(
      J_0(B_r r) - J_2(B_r r) \right) \right) \\
    \frac{1}{r} \frac{\partial}{\partial r} \left(r 
      \frac{\partial \phi}{\partial r} \right) &=
      -B_r \sin(B_z z) \left(\frac{1}{r} J_1(B_r r) + \frac{1}{2} B_r \left(
      J_0(B_r r) - J_2(B_r r) \right) \right)
  \end{align}
  Finally, inserting the expression for the Laplacian of the flux back into the
  diffusion equation.
  \begin{multline}
    D \left( B_r \sin(B_z z) \left( \frac{1}{r} J_1(B_r r) + \frac{1}{2} B_r
    \left( J_0(B_r r) - J_2(B_r r) \right) \right) + B_z^2 J_0(B_r r) \sin(B_z
    z) \right) + \Sigma_r J_0(B_r r) \sin(B_z z) = \\
    \frac{1}{\keff} \nu \Sigma_f J_0(B_r r) \sin(B_z z)
  \end{multline}
  Dividing through by $\sin(B_z z)$.
  \begin{multline}
    D \left( B_r \left( \frac{1}{r} J_1(B_r r) + \frac{1}{2} B_r
    \left( J_0(B_r r) - J_2(B_r r) \right) \right) + B_z^2 J_0(B_r r) \right)+
    \Sigma_r J_0(B_r r) = 
    \\\frac{1}{\keff} \nu \Sigma_f J_0(B_r r) 
  \end{multline}
  Dividing through by $J_0(B_r r)$ and expanding some terms.
  \begin{align}
    D \left( B_r \left( \frac{1}{r} \frac{J_1(B_r r)}{J_0(B_r r)} + 
      \frac{1}{2} \left(B_r - \frac{J_2(B_r r)}{J_0(B_r r)} \right) \right) 
      + B_z^2 \right)+ \Sigma_r &= \frac{1}{\keff} \nu \Sigma_f \\
    D \left( \frac{1}{r} B_r \frac{J_1(B_r r)}{J_0(B_r r)} + \frac{B_r^2}{2} -
      \frac{B_r^2}{2} \frac{J_2(B_r r)}{J_0(B_r r)} + B_z^2 \right) + \Sigma_r&=
      \frac{1}{\keff} \nu \Sigma_f  \\
    \frac{1}{r} B_r \frac{J_1(B_r r)}{J_0(B_r r)} + \frac{B_r^2}{2} -
      \frac{B_r^2}{2} \frac{J_2(B_r r)}{J_0(B_r r)} + B_z^2 + 
      \frac{\Sigma_r}{D} &= \frac{1}{\keff} \frac{\nu \Sigma_f}{D}\\
    \frac{1}{J_0(B_r r)} \frac{B_r^2}{2} \left(\frac{1}{r} \frac{2}{B_r} 
      J_1(B_r r) - J_2(B_r r) \right) + \frac{B_r^2}{2} + B_z^2 + 
      \frac{\Sigma_r}{D} &= \frac{1}{\keff} \frac{\nu \Sigma_f}{D}
  \end{align}
  Note the Bessel function recursion relationship.
  \begin{equation} \label{eq:bessel_recursion}
    J_{n+1}(\alpha r) + J_{n-1}(\alpha r) = \frac{2n}{\alpha r} J_n(\alpha r)
  \end{equation}
  Using \eref{eq:bessel_recursion} the term above is simplified.
  \begin{align}
    \frac{1}{J_0(B_r r)} \frac{B_r^2}{2} \left( J_0(B_r r) \right) + 
      \frac{B_r^2}{2} + B_z^2 + \Sigma_r &= \frac{1}{\keff} 
      \frac{\nu \Sigma_f}{D} \\
    B_r^2 + B_z^2 + \frac{\Sigma_r}{D} &= \frac{1}{\keff} \frac{\nu \Sigma_f}
      {D}
  \end{align}
  Now, an expression for the eigenvalue can be written.
  \begin{align}
    \keff &= \frac{\nu \Sigma_f}{D(B_r^2 + B_z^2) + \Sigma_r} \\
    &= \frac{\nu \Sigma_f}{D\left(\left(\frac{\alpha_0}{R}\right)^2 + 
      \left(\frac{\pi}{H}\right)^2 \right) + \Sigma_r} \\
    &= 0.996710620898177
  \end{align}
  NOTE: the shape of the boundary is extremely important for this  problem.
  Therefore, the mesh must be regenerated with a halved mesh parameter  $h$ for
  each refinement, rather than simply splitting nodes as splitting nodes would 
  not improve the description of the boundary.

